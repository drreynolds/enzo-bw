
\section{Introduction}
\label{sec:introduction}

The epoch of reionization (EoR) is a current frontier of cosmological research both observationally and theoretically. 
Observations constrain the transition from a largely neutral intergalactic medium (IGM) of primordial gas to a largely ionized one 
(singly ionized H and He) to the redshift interval $z \sim 11-6$, which is a span of roughly 500 Myr. 
The completion of H reionization by $z \approx 6$ is firmly established through quasar absorption
line studies to luminous, high redshift quasars which exhibit Ly $\alpha$ Gunn-Peterson absorption troughs \citep{FanCarilliKeating2006}. 
The precise onset of H reionization (presumably tied to the formation of the first luminous ionizing sources) is presently unknown observationally, however CMB measurements of the Thomson optical depth 
to the surface of last scattering by the WMAP and Planck satellites indicates that the IGM was substantially ionized by $z\sim 10$ %\cite{TracGnedin2011,WMAP7,Planck2013}.
{\bf \citep{Spergel2003,Komatsu2009,JarosikEtAl2011,Planck2013}. }
Since the optical depth measurement is redshift integrated  and averaged over the sky, the CMB observations provide no information about how reionization proceeded or the nature of the radiation sources that caused it.

 It is generally believed that reionization begins with the formation of Population III stars at $z \sim 20-30$ \citep{ABN02,Yoshida03,BL04,Sokasian04}, but that soon the ionizing photon budget becomes dominated by young, star forming galaxies (see e.g., \cite{Wise12,Xu13}), and to a lesser extent by the first quasars {\bf \citep{MadauEtAl1999, BoltonHaehnelt2007, HaardtMadau2012, BeckerBolton2013} }. Observations of galaxies in the redshift interval $6 \leq z \leq 10$ using the Hubble Space Telecope support the galaxy reionizer hypothesis, with the caveat that the faint end of the luminosity function which contributes substantially to the number of ionizing photons has not yet been measured \citep{Robertson10,Bouwens12}. 

Given the paucity of observational information about the {\em process} of cosmic reionization, researchers have resorted to theory and
numerical simulation to fill in the blanks. As reviewed by Trac \& Gnedin (2011), progress in this area has been dramatic, driven by a synergistic interplay between semi-analytic approaches and numerical simulations. The combination of these two approaches have converged on a qualitative picture of how H reionization proceeds assuming the primary ionizing sources are young, star-forming galaxies. The physics of the reionization process is determined by the physics of the source and sinks of ionizing radiation in an expanding universe. Adopting the $\Lambda$CDM model of structure formation, galaxies form hierarchically through the merger of dark matter halos. The structure and evolution of the dark matter density field is now well understood through ultra-high resolution numerical N-body simulations \citep{Millenium,Bolshoi} and through analytic models based on these simulations \citep{CooraySheth2002}. By making certain assumptions about how ionizing light traces mass and the dynamics of HII regions, a basic picture of the reionization process has emerged \citep {Furlanetto04,Furlanetto06,Iliev06,Zahn07} that is confirmed by detailed numerical simulations; e.g., \citep{Zahn11}.
%[summarize the basic picture]

The basic picture is that galaxies form in the peaks of the dark matter density field and drive expanding HII regions into their surroundings by
virtue of the UV radiation emitted from young, massive stars. These HII regions are initially isolated, but begin to merge into larger, Mpc-scale HII regions due to the clustering of the galaxy distribution (expansion phase). Driven by a steadily increasing global star formation rate and recombination time (due to cosmic expansion) this process goes on until HII regions completely fill the volume (overlap phase). In this picture, rare peaks in the density field ionize first while regions of lower density ionize later from local sources that themselves formed later. In this picture, referred to as ``inside-out reionization", void regions are the last to ionize because they have few local sources of ionization and remain neutral until an I-front from a denser region has swept over it. 

To date numerical simulations of reionization have fallen into two basic classes \citep{TracGnedin2011}: small-scale simulations that resolve the sources and sinks of ionizing radiation; and large-scale simulations that account for the diversity and clustering of sources. While ideally one would like to do both in a single simulation, this has not been feasible until now owing to numerical limitations. Historically, small-scale simulations came first. These simulations self-consistently modeled galaxy formation, radiative transfer, and photoionization/recombination within a hydrodynamic cosmological code. Comoving volumes were typically $\leq (10 Mpc)^3$ with spatial resolution of a few comoving kpc--sufficient to resolve high redshift dwarf galaxies and the baryonic cosmic web  \citep{Gnedin00a,RazoumovEtAl2002,RicottiEtAl2002,Petkova11,Finlator11}. Large-scale simulations followed, however these were not self-consistent radiation hydrodynamic cosmological simulations. Rather, density fields were simulated with a cosmological N-body or hydrodynamics code, and then ionization was computed in a post-processing step using a standalone radiative transfer code, typically a Monte Carlo or ray-tracing code
\citep{Ciardi01,Sokasian02,Sokasian03,Iliev06,Zahn07,TracCen2007,TracCenLoeb2008,ShinTracCen2008,Finlator09}.

The need to simulate large cosmological volumes coupled with the cost or limited scalability of available radiative transport methods led to a trend which continues to this day of using different numerical resolutions to model the N-body dynamics and the radiative transfer/ionization calculations. For example Iliev et al. (2006a) {\bf used a particle-mesh code to}
simulate N-body dynamics in a volume 100 Mpc/h on a side with a force resolution of 31 kpc/h, while performing the RT calculation on grids with comoving resolutions of 246 and 492 kpc/h. Similarly Trac \& Cen (2007,2008) and Shin, Trac \& Cen (2008) achieved a force resolution of 8.7 kpc/h in {\bf particle-mesh} N-body simulations in 50 and 100 Mpc/h boxes, but performed the RT calculation with a mesh resolution of 278 kpc/h. 
{\bf More recent simulations are typically far better resolved than that, using either P$^3$M (e.g. \cite{Iliev2012} or tree methods (e.g. \cite{PawlikSchaye2011, Petkova11}), which typically give 1-2 kpc/h or better force resolution. These references illustrate the trade-off between volume and resolution. The Iliev simulations are performed in boxes varying in size from 37 Mpc/h to 114 Mpc/h. Radiative transfer is performed on a mesh of 256$^3$ cells, regardless of the box size, resulting in a resolution mismatch ranging from 144 to 445. The other two references perform the radiative transfer calculation at the same resolution as the the undelying SPH simulation, however they have only been applied to small volumes.  } \st{When the RT is coarse-grained in this way, hydrodynamic effects are lost, and some corrections to the recombination rate due to small scale clumping must be made. This usually takes the form of a clumping factor correction evaluated either locally or globally using the higher resolution N-body data.} 

An interesting variant of the post-processing approach is the work of Trac, Cen \& Loeb (2008) who carried out a hydrodynamic cosmological simulation of reionization in a 100 Mpc/h box with $1536^3$ grid cells and particles, taking the emitting source population and subgrid clumping from a much higher resolution \st{($11,560^3$)} N-body simulation. This is an advance over previous work in that large scale baryonic flows are included self-consistently. However small scale radiation hydrodynamic effects, such as the retardation of I-fronts by minihalos \citep{Shapiro04} or the photoevaporation of gas from halos are not modeled. 

%What do we mean by direct numerical simulation?
In this paper we present a numerical method for simulating cosmological reionization in large cosmological volumes in which 
all relevant processes (dark matter dynamics, hydrodynamics, chemical ionization and recombination, radiation transport, local
star formation and feedback) are computed self-consistently on the same high resolution computational grid. 
\st{ We refer to this as direct numerical simulation, in analogy with turbulence simulations which solve the Navier-Stokes equations directly. We admit the analogy is not perfect, because not all physical scales are numerically resolved and we must employ a subgrid model for star formation. 
Nonetheless we use the term to connote that we solve the full set of dynamical and transport equations on a common discrete basis set (cells, particles). Another descriptor for our approach is} We refer to this as {\em resolution matched}, to distinguish our simulations from the fine/coarse dual resolution scheme used in previous large-scale simulations. 

The key numerical requirement for performing simulations that {\em both} resolve the sources and sinks of ionizing radiation {\em and} correctly model the abundance and clustering of sources is {\em algorithmic scalability}. Parallel scalability is also important, but of secondary importance to algorithmic scalability. Algorithmic scalability refers to how the time to solution scales with the number of unknowns $N$. Direct force evalulation gravitational N-body problems scale as $N^2$. While this is the most accurate approach, it is impractical for $N \sim 10^{10}$ which characterizes modern cosmological N-body simulations. Reionization simulations pose a similar scaling problem. If $N$ is the number of fluid elements (particles, cells) and $S$ is the number of ionizing sources, then the work scales as $N \times S$. At fixed resolution $S$ scales as $N$, since both are proportional to the volume simulated. If ray tracing is the method used for radiative transfer, and $R$ is the number of rays propagated per source, then the work scales as $N^2 R$. The factor $R$ is typically of order 100, but may be compensated for by the fact that $S/N \ll 1$. Therefore work scales as $N^2$ with the commonly used ray tracing approach, and this approach is not tenable for very large $N$. This is the underlying reason why previous large box simulations perform the radiative transfer calculation on a coarser grid than the dark matter calculation. For example, in the work of Trac \& Cen (2007), the disparity in scales is 32.

What is desired is an algorithm that is ideally $\mathcal O(N)$, but lacking that, no worse that $\mathcal O(N\log N)$. \st{Ray merging is one way to achieve this scaling with ray-based codes (e.g., Wise \& Abel 2011).} {\bf SPH-based methods \citep{PawlikSchaye2008, PawlikSchaye2011, Petkova11} inherit the scalability of the underlying SPH simulation. Others alleviate the scaling problem with adaptive rays, ray merging, short characteristics ray tracing or other techniques \citep{RazoumovCardall2005, Mellema2006, TracCen2007, ShinTracCen2008, WiseAbel11}. }  We have achieved $\mathcal O(N\log N)$ scaling by numerically representing the radiation field as a {\em grid field}, and employing optimally scalable geometric multigrid methods for the solution of the radiation field equation. In this work the radiation field is treated in the flux-limited diffusion (FLD) approximation, and discretized on the same grid as used for the dark matter and hydrodynamics. The method we describe below is currently implemented on uniform Cartesian grids within version 2 the community code  {\em Enzo} {\bf \citep{Enzo2014}}; an adaptive mesh version of this is under development and will be reported on in a forthcoming paper (Reynolds et al., {\em in prep}). 

In Sec. 2 we describe the mathematical formulation of the problem. In Sec. 3 we present the numerical method of solution, focusing on the solution of the coupled radiation diffusion, chemical ionization, and gas energy equations within the {\em Enzo} code framework. As {\em Enzo} has been described elsewhere, only a brief summary of its methods are included.  Section 4 contains results from verification tests (Sec. 4.1), validation tests (Sec. 4.2), parallel scaling tests (Sec. 4.3), and execution speed tests (Sec. 4.4).  We then illustrate the applicability of our method to cosmic reionization in Sec. 5, confining ourselves to a qualitative description of the results; a quantitative analysis of the results \st{will be} {\bf is} presented in \st{a forthcoming paper} {\bf \citep{So2014} }. We present a summary and conclusions in Sec. 6. 

