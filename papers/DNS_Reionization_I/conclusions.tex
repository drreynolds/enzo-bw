
\section{Summary and Conclusions}
\label{sec:conclusions}

%What we presented
We have described an extension of the {\em Enzo} code to enable the direct numerical simulation 
of inhomogeneous cosmological ionization in large cosmological volumes. By direct we mean
all dynamical, radiative, and chemical properties are solved self-consistently on the same mesh, 
as opposed to a postprocessing approach which coarse-grains the radiative transfer, as is done
in other works \citep{Iliev06,Zahn07,TracCen2007,TracCenLoeb2008,ShinTracCen2008,Finlator09}. 
Star formation and feedback are treated through a parameterized
phenomenological model, which is calibrated to observations. The goal of this work is to achieve a higher
level of self-consistency in modeling processes occuring outside the virial radii of luminous sources to  
better understand how recombinations in the clumpy intergalactic medium retard reionization and how
radiative feedback effects star formation in low mass galaxies. 

%Model limitations
In its current incarnation, the model has three principal limitiations. First, it is formulated on a fixed
Eulerian grid, which limits the spatial resolution that can be achieved. With a judicious choice of grid
sizes and resolutions, one can sample the dark matter halo mass function over a significant range of scales,
thereby including important sources and sinks of ionizing radiation. One can do a good job resolving the Jeans length 
in the diffuse IGM, which is important for ``Jeans smoothing" \citep{Gnedin00b}. However one cannot resolve the
internal structure of halos, which is important for calculating star formation rates and ionizing escape fractions. 
In this work we do not claim to be modeling these aspects self-consistently, but rather calibrate these unknown
parameters to observations. In Reynolds et al. (2013) we present the extention of our method to adaptive
mesh refinement (AMR), which directly addresses the numerical resolution limitation. 

The second model limitation is the use of FLD to model the transport of radiation, as opposed to a higher
order moment method such as OTVET \citep{GnedinAbel2001,Petkova09}. FLD has the well known deficiency of
not casting shadows behind opaque objects. However, as we have shown in Sec. \ref{sec:tests}, casting
shadows is not required to predict the evolution of the ionized volume fraction in a cosmological reionization
simulation, or to compute the photoevaporation time for an opaque cloud. Our {\em a priori} assumption that
small scale features like shadows will have little effect on large scale reionization processses are borne out
by these validation test. For simulating smaller scale processes where shadows may be important, such as the 
effect of halo substructure on the escape fraction of ionizing radiation, we note that our implicit
solution methodology is easily 
extended to higher-order moment methods \citep{HayesNorman2003,Petkova09}. 

The third model limitation is our simplified model for the radiation spectrum, which at the moment consists of 
monochromatic and grey with an assumed fixed SED. For simulating hydrogen reionization by soft UV radiation
from stellar sources
this spectral model is quite adequate compared to a multifrequency model (see RT09, Sec. 4.1). However for
harder radiation sources, such as Pop III stars and AGN, our model makes I-fronts that are too sharp, and does
not produce the preheating of gas ahead of the I-front by more penetrating, higher energy photons (``spectral
hardening"). The principal difference between our model and a multifrequency/multigroup model is in the temperature
distribution of the gas. Our model will slightly overpredict the temperature inside an HII region, and underpredict
the temperature outside of it, because all of the radiation energy is absorbed inside the I-front. Another way to
think about this is that in the multifrequency model in which the highest energy photons leak out of the HII region,
the characterisitic temperature of the radiation field inside the HII region is lower than outside of it. The standard
approach for dealing with the limitiations of our spectral model is to move to a multifrequency or multigroup 
discretization of the radiation field \citep{MirochaEtAl2012}. This is straightforward in practice, however the computational cost
increases linearly with the number of frequencies/groups. With the speed and memory of modern supercomputers
is this not a severe limitation, except for the very largest grids. Indeed we have implemented a multigroup FLD 
version of our method which is undergoing testing at the present time. 

%How well it works
Despite these limitations, the method is robust and acceptably fast. On verification tests for which analytic 
solutions are known, we have shown the method to be capable of high accuracy; the accuracy being
governed by grid resolution and the error tolerance parameter in the radiation diffusion calculation.
In validation tests, for which no analytic solution exists, we have shown that our method gives results which are 
qualitatively and quantitatively similar with those obtained with ray tracing and Monte Carlo methods 
\citep{IlievEtAl2006,IlievEtAl2009,Petkova09,WiseAbel11}, with what differences exist understood to be the result of the geometric
simplification of the radiation field inherent in FLD, and the difference in radiation spectrum modeling.
 
%How fast it is
Regarding the speed of our method, we have shown by direct comparison that a radiation hydrodynamic
simulation of cosmological reionization costs about $8 \times$ that of a corresponding pure hydrodynamic
model in which the IGM is ionized by a uniform UV background. We have not compared it to a postprocessing
radiative transfer code using ray tracing, although this would be a useful thing to do. Our method, which 
exhibits $\mathcal O(N\log N)$ scaling, should be competitive with, and possibly even beat ray tracing methods for 
very large numbers of sources.

%Potential for scaling up
Our method is highly scalable, as demonstrated in Sec. 4.3. This is due to two factors: (1) 
modeling radiation as a {\em field} instead of a collection of rays, which has no explicit dependence on the number
of point sources; and (2) the implementation of our radiation solver using optimally scalable parallel multigrid algorithms.
The application of our method to cosmological reionization by stellar sources is briefly discussed in Sec. 5 and 
graphically illustrated in Fig. 16.  The largest simulation we have completed to date with our method is identical to 
design, physics, and numerical resolution as that in Fig. 16, but in a volume 64 times as large (80 Mpc vs. 20 Mpc).
It was carried out on a 
mesh of size $3200^3$ using 46,875 compute cores on the Cray XT5 architecture {\em Jaguar}. 
Although the simulation cost many millions of core-hrs to compute, it is
the first simulation to cover the full range of halo masses thought to contribute to reionization and at the same time
modeling the gravitational, baryonic and radiative feedback processes self-consistently resolving the Jeans smoothing scale
in the ionized IGM. The results of this simulation will be presented in forthcoming papers (So et al. 2013a,b). 

This research was partially supported by National Science Foundation grants AST-0808184 and AST-1109243
and Department of Energy INCITE award AST025 to MLN and DRR. Simulations were performed on the {\em Kraken}
supercomputer operated for the Extreme Science and Engineering Discovery Environment (XSEDE)
by the National Institute for Computational Science (NICS), ORNL with support from XRAC allocation MCA-TG98N020 to MLN.
MLN, DRR and GS would like to especially acknowledge the tireless devotion to this project by our co-author Robert Harkness
who passed away shortly before this manuscript was completed. 


