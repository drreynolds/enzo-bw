
\section{Mathematical Formulation}
\label{sec:math_formulation}


%Primordial gas chemistry (do we need more detail here?) No, refer to Bryan et al. (2013)

We solve the coupled equations of multispecies gas dynamics, dark matter dynamics, self-gravity, 
primordial gas chemistry, radiative transfer, and gas cooling/heating in a comoving volume
of the expanding universe. In this paper we assume the governing equations are discretized on a cubic uniform Cartesian mesh  in comoving 
coordinates assuming periodic boundary conditions. In Reynolds et
al. (in prep.) we generalize our method to adaptive meshes. 
The background spacetime is assumed to be a FRW model with $\Lambda$CDM cosmological parameters \citep{WMAP7}. In this work we consider only the 5 ionic states of H and He and $e^-$; i.e., the commonly used ``6-species" model of primordial gas \citep{AbelEtAl1997, Anninos97}. Molecular hydrogen chemistry is ignored as we are primarily concerned with the later stages of H reionization driven by star formation in atomic line cooling galaxies. Star formation is modeled phenomenologically through a subgrid model described in the next section. Newly formed stars are sources of UV radiation, and the radiation is transported in the grey flux-limited diffusion approximation. Star formation in spatially distributed galaxies thus sources an inhomogeneous and evolving ionizing radiation field, which is used to calculate the local ionization and thermal state of the gas. This in turn controls the local cooling rate of the gas, and by virtue of the subgrid star formation model, the local star formation rate. We thus have a closed system of equations that we can evolve forward in time subject to the choice of initial conditions. In all but the verification test problems, cosmological initial conditions are generated using standard methods.  

The choice of flux-limited diffusion (FLD) is motivated by its
simplicity and its ability to smoothly transition between optically
thin and thick regimes. Its properties as well as its limitations are
well understood, and efficient numerical methods exist for parallel
computation (e.g., Hayes et al. 2006). A second motivation is that we
are interested in modeling reionization in large cosmological volumes
and field-based solvers scale independently of the number of sources, unlike ray tracing
methods. In the early stages of reionization, when HII regions are
largely isolated, FLD provides accurate I-front speeds, as shown by
our verification tests in Sec. 4.1. At late times, during and after
overlap, the gas is bathed in a diffuse radiation field arising from
numerous point sources for which the angular structure of the
radiation field is unimportant. It is during the early percolation
phase when several HII regions merge that FLD is inaccurate with
regard to the angular distribution of the radiation field. This leads
to some inaccuracies of the shapes of the I-fronts compared to a
solution obtained using ray tracing (see Sec. 4.3). However we
consider these shape differences of secondary importance since we are
interested in globally averaged ionization properties. A well known
limitation of FLD is that opaque blobs do not cast shadows if they are illuminated from one side (e.g.,
Hayes \& Norman 2003). Instead, the radiation flows around the
backside of the irradiated blob. By contrast, a ray tracing method
will cast a sharp shadow (Iliev et al. 2009, Wise \& Abel 2011). What
matters for global reionization simulations however is how long the
opaque blobs remain self-shielded; i.e., their photoevaporation
times. We have compared the photoevaporation times for identically
resolved blobs using FLD and the ray tracing method of Wise \& Abel
(2011), and find them comparable despite the inability of FLD to cast
a shadow (sec Sec.~\ref{subsubsec:test7}). 

{\bf Finally we comment on the use of a {\em grey} treatment of the radiation field. 
Grey FLD is an improvement over monochromatic radiative transfer as it provides a formalism for calculating the contributions of higher energy photons above the ionization threshold to the frequency-integrated photoionization rate and photoheating rate. It is not as accurate as multifrequency/multigroup radiative transfer in that it does not model spectral hardening of the radiation field and preionization ahead of the I-front. At present, no large-scale simulations of cosmological reionization include these effects, which are expected to be small for the soft ionizing spectrum of Pop II stars. }



We consider the coupled system of partial differential equations
\citep{ReynoldsHayesPaschosNorman2009},
\begin{align}
  \label{eq:gravity}
  \nabla^2 \phi &= \frac{4\pi g}{a}(\rhob + \rho_{dm} - \langle \rho \rangle), \\
  \label{eq:cons_mass}
  \partial_t \rhob + \frac1a \vb \cdot \nabla
    \rhob &= -\frac1a \rhob \nabla\cdot\vb - \dot{\rho}_{SF}, \\
  \label{eq:cons_momentum}
  \partial_t \vb + \frac1a\(\vb\cdot\nabla\)\vb &=
    -\frac{\dot{a}}{a}\vb - \frac{1}{a\rhob}\nabla p - \frac1a
    \nabla\phi, \\
  \label{eq:cons_energy}
  \partial_t e + \frac1a\vb\cdot\nabla e &=
    - \frac{2\dot{a}}{a}e
    - \frac{1}{a\rhob}\nabla\cdot\left(p\vb\right) 
    - \frac1a\vb\cdot\nabla\phi + G - \Lambda  + \dot{e}_{SF} \\
  \label{eq:chemical_ionization}
  \partial_t \mn_i + \frac{1}{a}\nabla\cdot\(\mn_i\vb\) &=
    \alpha_{i,j} \mn_e \mn_j - \mn_i \Gamma_{i}^{ph}, \qquad
    i=1,\ldots,N_s \\
  \label{eq:cons_radiation}
  \partial_t E + \frac1a \nabla\cdot\(E \vb\) &= 
    \nabla\cdot\(D\nabla E\) - \frac{\dot{a}}{a}E - c \kappa E + \eta.
\end{align}
The comoving form of Poisson's equation \eqref{eq:gravity} is used to
determine the modified gravitational potential, $\phi$, 
where $g$ is the gravitational constant, $\rhob$ is the comoving
baryonic density, $\rho_{dm}$ is the dark matter density, and 
$\langle \rho \rangle$ is the cosmic mean density.  The collisionless
dark matter density $\rho_{dm}$ is evolved using the Particle-Mesh
method, as described in
\cite{HockneyEastwood1988,NormanBryan1999,OSheaEtAl2004}.  The
conservation equations \eqref{eq:cons_mass}-\eqref{eq:cons_energy}
correspond to the compressible Euler equations in a comoving
coordinate system \cite{BryanEtAl1995}.  These relate the density to
the proper peculiar baryonic velocity $\vb\equiv a(t)\dot{\xvec}$, the
proper pressure $p$, and the total gas energy per unit mass $e$.   
The equations \eqref{eq:chemical_ionization} model
ionizaton processes between the chemical species HI, HII, HeI, HeII,
HeIII and the electron density.  Here, $\mn_i$ denotes the
$i^{th}$ comoving elemental species number density, $\mn_e$ is the
electron number density, $\mn_j$ corresponds to ions that react
with the species $i$, and $\alpha_{i,j}$ are the reaction rate
coefficients defining these interactions
\citep{AbelEtAl1997,HuiGnedin1997}.  The equation
\eqref{eq:cons_radiation} describes the flux-limited diffusion (FLD)  
approximation of radiation transport in a cosmological medium
\citep{HayesNorman2003,Paschos2005}, where $E$ is the comoving grey
radiation energy density.  Within this equation, the function
$D$ is the {\em flux limiter} that depends on face-centered values of
$E$, $\nabla E$ and the opacity $\kappa$ \citep{Morel2000},
\begin{align}
  D &= \min\left\{c \(9\kappa^2 + R^2\)^{-1/2}, D_{max}\right\},\quad\mbox{and}\quad
  R = \max\left\{\frac{|\partial_{x} E|}{E},R_{min}\right\}.
\end{align}
Here the spatial derivative within $R$ is computed using
non-dimensional units at the computational face adjoining two
neighboring finite-volume cells, $D_{max}=0.006\,c\,L_{unit}$ and
$R_{min}=10^{-20}/L_{unit}$ with $L_{unit}$ the length
non-dimensionalization factor for the simulation, and the 
face-centered radiation energy density and opacity are computed using
the arithmetic and harmonic means, respectively,
\[
   E = \frac{E_1 + E_2}{2}, \qquad
   \kappa = \frac{2\kappa_1 \kappa_2}{\kappa_1 + \kappa_2}.
\]
Among the many available limiter formulations we have tested
(\citep{HayesNorman2003,Morel2000,ReynoldsHayesPaschosNorman2009}), this 
version performs best at producing causal radiation propagation
speeds in the low-opacity limit typical of the late stages of reionization simulations.  


Cosmic expansion for a smooth homogeneous background is modeled by the
function $a(t)\equiv(1+z)^{-1}$ 
, where the redshift $z$ is a function
of time. $a(t)$ is obtained from a solution of the Friedmann equation
for the adopted cosmological parameters. All comoving densities $\rho_i$ relate to the proper
densities through $\rho_i \equiv \rho_{i,\text{proper}}a(t)^3$. All
spatial derivatives are taken with respect to the comoving position
$\xvec\equiv{\bf r}/a(t)$.  We use a standard ideal gas equation of
state to close the system,
\begin{equation}
\label{eq:eos}
  e = \frac{p}{2 \rhob/3} + \frac12|\vb|^2.
\end{equation}




\subsection{Model Coupling}
\label{subsec:equation_coupling}

The equations \eqref{eq:gravity}-\eqref{eq:cons_radiation} are coupled
through a variety of physical processes.   In defining our grey
radiation energy density $E$, we allow specification of an assumed
spectral energy distribution (SED), $\chi_E(\nu)$.  Here, we write
the frequency-dependent radiation density using the
decomposition $\Enu(\xvec,t,\nu)=\tilde{E}(\xvec,t)\,\chi_E(\nu)$.  This
relates to the grey radiation energy density $E$ through the equation
\begin{equation}
\label{eq:grey_definition}
  E(\xvec,t) = \int_{\nu_1}^{\infty} \Enu(\xvec,t,\nu)\,\mathrm d\nu =
  \tilde{E}(\xvec,t) \int_{\nu_1}^{\infty} \chi_E(\nu)\,\mathrm d\nu,
\end{equation}
where $\tilde{E}$ is an intermediate quantity that is never computed.
We note that this relationship is valid only if the indefinite
integral of $\chi_E(\nu)$ exists, as is the case for quasar and
stellar type spectra.  Implemented in {\em Enzo} are a variety of user-selectable
SEDs including black body, monochromatic, and powerlaw (some of these
are used for the verification tests; see Sec. 4.2). 
In our application to cosmic reionization, we utilize the SED for low
metallicity Pop II stars from \cite{RicottiEtAl2002}.

With this in place, we define the radiation-dependent photoheating
and photoionization rates \citep{Osterbrock1989},
\begin{align}
  \label{eq:photoheating}
  G &= \frac{c E}{\rho_b} \sum_i^{N_s} \mn_i \left[\int_{\nu_i}^{\infty}
    \sigma_i(\nu) \chi_E(\nu)\left(1-\frac{\nu_i}{\nu}\right)\,\mathrm
    d\nu\right]   \bigg /
  \left[\int_{\nu_1}^{\infty} \chi_E(\nu)\,\mathrm d\nu\right], \\
  \label{eq:photoionization}
  \Gamma_i^{ph} &= \frac{c E}{h} \left[\int_{\nu_i}^{\infty}
    \frac{\sigma_i(\nu) \chi_E(\nu)}{\nu}\,\mathrm d\nu \right]  \bigg /
  \left[\int_{\nu_1}^{\infty} \chi_E(\nu)\,\mathrm d\nu\right].
\end{align}
Here, $\sigma_i(\nu)$ is the ionization cross section for the species 
$\mn_i$, $h$ is Planck's constant, and $\nu_i$ is the frequency
ionization threshold for species $\mn_i$ ($h\nu_{HI}$ = 13.6 eV, 
$h\nu_{HeI}$ = 24.6 eV, $h\nu_{HeII}$ = 54.4 eV).

In addition, gas cooling due to chemical processes occurs through the
rate $\Lambda$ that depends on both the chemical number densities 
and current gas temperature \citep{AbelEtAl1997,Anninos97},
\begin{equation}
\label{eq:temperature}
  T = \frac{2\, p\,\mu\, m_p}{3\, \rhob\, k_b},
\end{equation}
where $m_p$ corresponds to the mass of a proton, $\mu$ corresponds to
the local molecular weight, and $k_b$ is Boltzmann's constant.
In addition, the reaction rates $\alpha_{i,j}$ are highly temperature
dependent \citep{AbelEtAl1997,HuiGnedin1997}. 
The opacity $\kappa$ depends on the local ionization states $\mn_i$
and the assumed SED $\chi_E$, 
\begin{align}
  \label{eq:opacity}
  \kappa &= \sum_{i=1}^{N_s} \mn_i \left[ \int_{\nu_i}^{\infty}
    \sigma_i(\nu) \chi_E(\nu)\,\mathrm d\nu \right]  \bigg /
  \left[\int_{\nu_i}^{\infty} \chi_E(\nu)\,\mathrm d\nu\right].
\end{align}
The emissivity $\eta$ is based on a star-formation ``recipe'' described below. 


