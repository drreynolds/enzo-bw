Our current picture of the mechanics of reionization come mainly from theoretical and computational investigations. Within the past decade a picture has come into focus by combining the known physics of structure formation in a cold dark matter universe with
ionization front physics \citep{CiardiEtAl2003,IlievEtAl2006,TracGnedin2011}. This picture is referred to
as ``inside-out" reionization, and there is good agreement on its basic
features from both detailed numerical simulations and semi-analytic models.
The name Inside-out refers to the fact that within this picture, ionization fronts propagate
outward from the densest regions of the universe where the first ionizing
sources form to less dense regions and finally to the underdense void
regions far from radiation sources.
Topologically there are three distinguishable phases: (1) early
growth of isolated H{\footnotesize II} regions; (2) percolation of H{\footnotesize II} regions as more sources
turn on and cluster, and (3) complete overlap of H{\footnotesize II} regions. As reviewed by
\citep{TracGnedin2011}, the details of this depend critically
on the  detailed properties of the sources and sinks of ionizing radiation which are
not well characterized observationally. 

Many of the phenomena described above cannot be observed directly
using the current generation of instruments, therefore scientists
employ computer simulations to determine many of these details, hoping
that future telescopes will be able to confirm their findings.  Due to
the high computational cost of doing fully self-consistent 3D
simulations of the multi-physics phenomena, there have been many
efforts to approximate and simplify these complicated processes.
% Cite 21 cm, JWST, future telescopes here!!!

Around the turn of the century, there arose many semi-analytical 
attempts to determine some general features of the EoR  
\citep{ValageasSilk1999, MadauEtAl1999, MiraldaEscudeEtAl2000}.  
These work often use the clumping factor (can be taken from
cosmological simulations e.g. \citealt{GnedinOstriker1997}) to
estimate the recombinations in dense clumps of gas that the radiation
must overcome in order to keep the region ionized.  They find that if
emissivity does not increase with redshift for z $>$ 4, then the
Gunn-Peterson trough should be found at z $\simeq$ 6.  

Others soon followed this methodology; works such as 
\citep{CiardiEtAl2003, IlievEtAl2006, ZahnEtAl2007} used N-body
simulations of pure dark matter to locate where these overdense regions
should be.  Some of them also assumed that baryons follow dark matter on those
scales ($\sim$100 Mpc/h), determine what
the resulting clumping factor should be, and then calculate the
radiation requirement to ionize the entire Universe in a post-process
step.  While the above method is generally established, others
(e.g. \citealt{IlievEtAl2006, TracCen2007}) add more detailed physics
of radiation transport by using ray-tracing or adaptive ray-tracing
methods.  Some find that complete overlap of the ionizing bubbles
occurs around z $\simeq$ 11, consistent with WMAP data at the 
time, while others find an overlap at z $\simeq$ 6 to 6.5.  

Still, there are studies such as \citep{PawlikEtAl2009, RaicevicTheuns2011} 
that attempt to improve upon the accuracy of recombination count based on the clumping
factor.  The latter use a local clumping factor to obtain a better estimate
of how ``clumpy'' the dark matter is in the region, and are able to
get an improved estimate of the true recombination rate.  Additionally, a trend towards
more realistic and self consistent treatment of dark matter and
baryonic matter in fully 3D simulations emerged such as \citep{PetkovaSpringel2011a, 
PetkovaSpringel2011b}. 

%stated what they did but not what they found, expand on first paragraph, summarizing what they %found

In this work, we propose to use a similar implementation as 
\citealt{PetkovaSpringel2011b}, by adopting a Flux Limited Diffusion
(FLD) approximation of radiation transport processes.  Without 
assuming that baryons are following dark matter, we couple our FLD
radiation solver to models that separately calculate baryonic and dark
matter dynamics.  In this way, interactions between dark matter,
baryons and radiation are all coupled together consistently
within the calculation, instead of post-processing the radiation transport.

Under the FLD approximation, we do not need to solve the exact
expensive radiation transport equation, which remain computationally
infeasible, and we are not limited by the number of radiating sources
in the simulation, a limitation for ray-tracing methods.  With the
tools above, we find that the clumping factor is actually a poor
predictor for the time when the universe is ionized, due to the various
definitions of clumping factor in the literature, and that its use
usually neglects radiation effects on the distribution of matter.  By
consistently simulating radiation hydrodynamics, we are able to
directly calculate the time when the universe is ionized and further
explore the details of reionization in a more accurate manner. 
