\section{Introduction}
\label{sec:introduction}

The Epoch of Reionization (EoR) is an active area of research observationally,
theoretically, and computationally. Observations constrain the tail end of hydrogen reionization
to the redshift range $z=6-8$ \citep{RobertsonEtAl2010}. These observations include the presence of Gunn-Peterson
troughs in the Ly $\alpha$ absorption spectra of high redshift quasars \citep{FanEtAl2006}, and 
the strong evolution of Lyman $\alpha$ emitter luminosity function (Robertson et al. 2010 and references
therein.) 
Observations from the WMAP and Planck satellites tell us that the universe was substantially 
ionized by $z \approx 10$ but can say little about the
reionization history or topology \citep{JarosikEtAl2011,Planck2013}. 
High redshift 21cm observations hold forth great promise of elucidating the details of this transition \citep{BarkanaLoeb2007, PritchardLoeb2012}, but these results are still in the future. 

It is believed that early star forming galaxies provided the bulk of the UV photons responsible for
reionization \citep{RobertsonEtAl2010,RobertsonEtAl2013}, but early QSOs may have also contributed \citep{MadauEtAl1999, BoltonHaehnelt2007, HaardtMadau2012}.  The ``galaxy reionizer" hypothesis has been greatly strengthened by the recent advances in the study of high redshift galaxies afforded by the IR-sensitive Wide Field Camera 3 (WFC3) aboard the Hubble Space Telescope \citep[e.g.][]{RobertsonEtAl2010, RobertsonEtAl2013, BouwensEtAl2011, BouwensEtAl2011b, OeschEtAl2013}.  Within uncertainties, the luminosity function of $z=6$ Lyman break galaxies (LBGs) appears to be sufficient to account for reionization at that redshift from a photon counting argument \citep{BoltonHaehnelt2007, RobertsonEtAl2010, BouwensEtAl2012}. Among the observational uncertainties are the faint-end slope of the galaxy luminosity function \citep{WiseCen2009,LabbeEtAl2010,BouwensEtAl2012}, the spectral energy distribution of the stellar population \citep{CowieEtAl2009,WillotEtAl2010,HaardtMadau2012}, and the escape fraction of ionizing photons \citep{WyitheEtAl2010, YajimaEtAl2011, MitraEtAl2013}. Among the theoretical uncertainties are the number of ionizing photons per H atom required to bring the neutral IGM to its highly ionized state by $z=6$, the clumping factor correction to the mean IGM recombination time \citep{PawlikEtAl2009, RaicevicTheuns2011, FinlatorEtAl2012, ShullEtAl2012, RobertsonEtAl2013}, 
and the contribution of Pop III stars and accreting black holes to the early and late stages of reionization \citep{BoltonHaehnelt2007,TracGnedin2011,AhnEtAl2012}. 

When assessing whether an observed population of high-z galaxies is capable of 
reionizing the universe (e.g., Robertson et al. 2013), observers often use the criterion derived
by \cite{MadauEtAl1999} for the ionzing photon volume density $\dot{\mathcal{N}}_{ion}$ necessary
to maintain the clumpy IGM in an ionized state:

\begin{align}
\label{eq:ndot}
\dot{\mathcal{N}}_{ion}(z) &= \frac{\bar{n}_\mathrm{H}(0)}{\bar{t}_{rec}(z)} = (10^{51.2}s^{-1}Mpc^{-3})\left(\frac{C}{30}\right) \notag\\
&\times \left(\frac{1+z}{6}\right)^3 \left(\frac{\Omega_b h_{50}^2}{0.08}\right)^2,
\end{align}

where $\bar{n}_\mathrm{H}(0)$ is the mean comoving number density of H atoms, $C \equiv \langle n^2_\mathrm{H\,II} \rangle /\langle n_\mathrm{H\,II} \rangle ^2$ is the H {\footnotesize II} clumping factor (angle brackets denote volume average over a suitably large volume that the average is globally meaningful),
and the rest of the symbols have their usual meaning. 
The origin of this formula is a simple photon counting argument, which says 
that in order to maintain ionization at a given redshift $z$, 
the number of ionizing photons emitted in
a large volume of the universe multiplied by a characteristic recombination
time, denoted $\bar{t}_{rec}$,  must equal the number of hydrogen atoms: $\dot{\mathcal{N}}_{ion} \times \bar{t}_{rec} = \bar{n}_\mathrm{H}(0)$.
The clumping factor enters as a correction factor to account for the density inhomogeneties in the IGM
induced by structure formation. We note that $\bar{t}_{rec}$ is not the volume average of the local recombination time of the
ionized plasma, as this would heavily weight regions with the {\em longest} recombination
times; i.e. voids. A proper derivation of  Equation \eqref{eq:ndot} shows that $\bar{t}_{rec} \propto \langle t_{rec}^{-1} \rangle ^{-1}$, which weights regions
with the {\em shortest} recombination times; i.e. regions at the mean density and above. 

Equation \eqref{eq:ndot} is based on a number of simplifying assumptions discussed by
\cite{MadauEtAl1999}, including the assumption $\bar{t}_{rec} \ll t$. It is this assumption that allows history-dependent effects to be ignored, and a quasi-instantaneous analysis of the 
photon budget for reionization to be done. The validity of this assumption is naturally
redshift dependent, but it is also dependent upon the adopted definition of $\bar{t}_{rec}$.  A second comment about Equation \eqref{eq:ndot} is that it does not ask how many ionizing photons per H atom are required to convert a neutral IGM to a fully ionized one, only how many are required to {\em maintain} the IGM in an ionized state. Because the recombination time is short at high redshifts, it is expected that this number is greater than one. 

In this paper we examine these and related topics within the context of a direct numerical simulation of cosmic reionization based on a new flux-limited diffusion radiation transport solver installed in the {\em Enzo} code \citep{NormanEtAl2013} (hereafter Paper I).
Our approach self-consistently couples all the relevant physical processes 
(gas dynamics, dark matter dynamics, self-gravity, star formation/feedback, 
radiative transfer, nonequilibrium ionization/recombination, heating and cooling) and evolves the
system of coupled equations on the same high resolution mesh. We refer to this
approach as {\em direct numerical simulation} or {\em resolution matched}, in contrast to previous approaches 
which decouple and coarse-grain the radiative transfer and ionization balance 
calculations relative to the underlying dynamical calculation.
Our method is
scalable with respect to the number of radiation sources, size of the mesh, and the
number of computer processors employed.
This scalability permits us to simulate cosmological reionization in large cosmological
volumes (L $\sim100$ Mpc) while directly modeling the sources and sinks of ionizing 
radiation,  
including radiative feedback effects such as photoevaporation of gas from halos, 
Jeans smoothing of the IGM, and enhanced recombination due to small scale clumping. 
In this the first of several application papers, we investigate in a volume of modest size (L=$20$ Mpc) the mechanics of reionization from stellar sources forming in high-$z$ galaxies, the role of gas clumping, recombinations, and the photon budget required to complete reionization.

By analyzing this simulation we are able to critically examine the validity of  Equation \eqref{eq:ndot} as a predictor of when the end of EoR will occur, and we can calculate the integrated number of ionizing photons 
per H atom needed to ionize the simulated volume $\gamma_{ion}/H=\int dt \dot{\mathcal{N}}_{ion} / \bar{n}_\mathrm{H}(0)$. Ignoring recombinations within the virial radii of collapsed halos, we find $\gamma_{ion}/H \approx 2$. This result
supports the ``photon starved'' reionization scenario discussed by \cite{BoltonHaehnelt2007}.
We also examine whether modern revisions to Equation \eqref{eq:ndot} using alternatively defined 
clumping factors \citep{PawlikEtAl2009, RaicevicTheuns2011, FinlatorEtAl2012, ShullEtAl2012} are 
improvements over the original. We find they systematically overestimate the redshift of reionization completion $z_{reion}$ because the condition $\bar{t}_{rec}/t \ll 1$ is never obeyed.  We study the accuracy and validity of the time-dependent analytic model of \cite{MadauEtAl1999}, and find that while it is in better agreement with the simulation, it also overestimates $z_{reion}$ because it ignores important corrections to the ionization term at early and late times.

This paper is organized as follows: in \S\ref{Method} we discuss the design criteria 
for the simulation and briefly outline the basic equations and implementation of the FLD radiation transport model, referring the
reader to Paper I for a more complete description of the numerical algorithms and tests.  
In \S\ref{GeneralResults},  we present some general features of the simulation and demonstrate its broad consistency with observed star formation rate density and high redshift 
galaxy luminosity function. 
%Because we track the ionization fraction of the gas at every point in the volume as a function of time, a more quantitative language is required to say when the simulated volume is X\% ionized; this is provided in \S\ref{QuantitativeLanguage}. 
%In \S\ref{IOOI} we examine the mechanics of reionization in our simulation, and in particular question whether ``inside-out'' or ``outside-in'' is a better description of certain phases of evolution. 
In \S\ref{sec:ClumpingFactors} we examine the accuracy of different clumping factor approaches to estimating the redshift of complete reionization.
In \S\ref{escape} we derive a global estimate for the circumgalactic absorption of ionizing radiation from our simulation. 
In \S\ref{Qdot} we test a simple analytic model for the evolution of the ionized volume fraction $Q_\mathrm{H\,II}$ and present an improvement to the model which better agrees with our simulation. 
In \S\ref{Discussion} we
discuss implications of our results on the current understanding of
reionization.  And finally, in \S\ref{Conclusions} we end with a
summary of our main results and conclusions.


