
\section{Introduction}
\label{sec:introduction}

In the post Big Bang era, the universe is filled with mostly neutral Hydrogen (H {\footnotesize I}) with some traces of neuetral Helium (He {\footnotesize I}).  Due to the presence of electrons on these neutral atoms, the universe remained opaque to radiation.  When the first generation of stars (Pop {\footnotesize III}) formed, radiation emitted by them are absorbed by the surrounding neutral atoms, bumping their electrons to a higher energy level, or if the energy was sufficient, ionize the atoms by knocking the electrons free from the neuclei altogether.  With the electrons freed from the neuclei of atoms that form the intergalactic medium (IGM) which was mostly Hydrogen, radiation can now penetrate deeper into the IGM. Having ionized more of the IGM, the radiation from stars causes bigger and bigger bubbles of ionized atoms.  Finally, these bubbles which are virtually transparent to radiation, grow over time, until they overlap and fill the entire Universe.  With the Universe mostly transparent, radiation emanating from different sources can then reach us, enabling us to see stars along with other celestial objects in the sky.

After scientists' attention shift to doing 3D N-body simulations, people want to model and simulate the previously mentioned epoch of reionization.  However, because of the high computation cost of doing fully self-consistent 3D simulation of the multi-physics phenomena, there have been many efforts to approximate and simplify the complicated equations.  Around the turn of the century, semi-analytical attempts such as \citep{MadauHaardtRees1999, MiraldaEscudeHaehneltRees2000}, tries to get some general features of the epoch of reionization.  They do so by using the clumping factor to estimate the recombinations in dense clump of gas that the radiation has to overcome in order to keep the region ionized.  They find that if emissivity does not increase with redshift for z $>$ 4, then the Gunn-Peterson trough should be found at z $\simeq$ 6.  Others soon follow the methodology.  Works mentioned in the review article \citep{LoebBarkana2001} use N-body simulations of pure dark matter to locate where the overdense regions should be.   They also assume that baryons follow dark matter on those scales, determine what the clumping factor should be, then calculate the radiation requirement to ionize the entire Universe in a post-process step.  While the above method is generally established, others such as \citep{IlievEtal2006} add more detailed physics of radiation transport by using the ray-tracing method.  They, however, find that complete overlap of the ionizing bubbles occurs around z $\simeq$ 11, consistent with WMAP data at the time.  Still, there are people such as \citep{RaicevicTheuns2011} trying to improve upon the accuracy of recombination count based on the clumping factor.  They use a local clumping factor to get a better estimate of how clumpy the dark matter is in the region, and are able to get a better estimate of the recombination.  Additionally, a trend towards more realistic and self consistent treatment of dark matter and baryonic matter emerged such as \citep{ZahnEtal2007,PetkovaSpringel2011}.

%stated what they did but not what they found, expand on first paragraph, summarizing what they %found

We are hereby proposing to use a similar implementation as \citep{PetkovaSpringel2011}, by adopting Flux Limited Diffusion (FLD) to approximate radiation transport, but with Enzo instead of GADGET as the code base.  Instead of assuming that baryons follow dark matter like many before, we now calculate the radiation hydrodynamics with the FLD module, coupling the the two forms of matter together.  In this way, the interaction between dark matter, baryons and the presence of radiation are all coupled together during calculation consistently, instead of being separate effects.  With the FLD apprimation, we do not have to solve the exact expensive radiation transport equation which is still computationally infeasible, but we are not limited by the amount of radiating sources that can exist in the simulation, a limitation for simulations utilizing the ray-tracing method.  With the tools above, we find that the clumping factor is actually a poor predictor for the time when the whole universe is ionized, due to the way the clumping factor has been defined and used, and that radiation affects the distribution of matter.  By fully simulating radiation hydrodynamics, we can directly calculate the time when the universe is ionized and explore further the details of reionization more accurately.

In \S\ref{Method} of the paper, I will very briefly outline the basic equations and implementation of the radiation transport, referring the reader to our method paper and earlier works for a more detailed discussion and description of the underlying machinery of Enzo and the new radiation solver.  Then in \S\ref{Results}, I will present the results we have obtained from the simulation we named BSM10RAD800.  In \S\ref{Discussion} I will discuss implications of the results on our field.  And finally, in \S\ref{Conclusions} I will end with summary and conclusion.
