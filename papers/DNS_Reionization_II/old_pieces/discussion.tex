
\section{Discussion}
\label{Discussion}

% circle back to listuerature, say if results agree, if so how, why that might be.  Raicevic Theuns, Folker
% Springel <2010+>, Finaltor paper address the similiarities/differences.  At the results level identify the 
% relavant papers.  Trac Gnedin review 2009

In the literature, most simulation with radiation transfer implemented resulte in an ionized universe by redshift of around 6, in agreement with the Gunn-Peterson trough (cite papers).  Using our model of FLD Radiation Transfer, we find in our fiducial model that 99.9% of the universe can be Fully ionized simply by PopII stars by redshift of around 5.5 (can I say in agreement with the "observation/theory" community?).  We also find that although the star formation rate density is within the error of (cite Bowens et al 2011), the numbers are approaching the higher end.  Our simulation's self consistent radiation transfer allow us to discuss about the number of photons emitted without assumptions of the environment and approximations (cite clumping factor papers).  This lets us investigate the photon to baryon ratio for a specific level of ionization for the universe.

We identified a couple quantitative ways to think about the epoch of reionization.  Mainly, when talking about the ionization state of the universe, we should specify whether the Hydrogen is 10% (Ionized), 1e-3 (Well Ionized), or 1e-5 (Fully Ionized).  Secondly, when talking about the ionization volume fraction, we also want to specify the percentages of the simulated volume.  Often, there seems to be tiny patchy regions that are low in Hydrogen ionization level so it is hard to achieve 100.0% of the volume.  In terms of the Equation \ref{Ndot}. we find that the H {\footnotesize II} clumping factor may be a good approximator at predicting roughly when the universe reaches Well Ionized level, but to get a precise redshift, clumping factor based prediction does not seem to be well suited for the task.  The clumping factor has always been viewed as a way to approximate the recombination rate.  So it is a way to estimate the results from running expensive fully consistent radiation transfer simulations, but not a replacement (cite Raicivic, Theuns).  Their efforts of thresholding the clumping factor and using a local version for precision may be overshadowed by the fact that the simulation used had many assumptions about the bahavior of H {\footnotesize II}.  We demonstrated that by using variations of the definition of the clumping factor, we can get results that vary by a wide range.  Although some resulted in a redshift of reionization that is consistent, the range of results (underscore/undermine?) its (precision/trustworthiness?). 

Kim Griest: use stronger/defininite language
Matt: don't use too strong of language
May need update on photon budget from Finlator
Risa Wechsler suggest to come up with a correction for C based estimate
cite YT, mention XSEDE resources
cite where eq2 comes from and what they assumed
calibrate N dot for both BSM800 and SED800
compare contrast with http://arxiv.org/abs/1008.4459

{\bf [Mike]}

