\section{Discussion}
\label{Discussion}

%\begin{itemize}
%\item what we have done
\subsection{Significance of our Main Results}
We have carried out a fully-coupled radiation hydrodynamic cosmological simulation of hydrogen reionization by stellar sources using an efficient flux-limited diffusion radiation transport solver coupled to the Enzo code (Paper I). This method has the virtue of a high degree of scalability with respect to the number of sources, which allows us to simulate reionization in large cosmological volumes including hydrodynamic and radiative feedback effects self-consistently. In this paper we have presented first results from a simulation in a cosmological volume of modest size--20 Mpc comoving--to investigate the detailed radiative transfer, nonequilibrium photoionization, photoheating and recombination processes that operate during reionization and dictate its progress. In a future paper we apply our method to larger volumes to examine the large scale structure of reionization, evolution of the bubble size distribution, etc. 

The simulation presented here is carried out on a uniform mesh of $800^3$ cells and with an equivalent number of dark matter particles. As such, the mass resolution is sufficiently high to evolve a dark matter halo population which is complete down to ($M_{halo} \approx 10^8 M_{\odot}$) which cools via H and He atomic lines. However, a spatial resolution of 25 kpc comoving poorly resolves internal processes within early galaxies, but does an excellent job of resolving the Jeans length in the photoionized IGM \citep{BryanEtAl1999}. Our simulation is most appropriately thought of as a high redshift IGM simulation which evolves an inhomogeneous ionizing radiation field sourced by star-forming early galaxies. Star formation is modeled using a modified version of the Cen \& Ostriker (1992) recipe that can be tuned to reproduce the observed star formation rate density (SFRD) \citep{SmithEtAl2011}. We have tuned our simulation to roughly match the observed SFRD \citep{BouwensEtAl2011,RobertsonEtAl2013} for $z\geq 7$, but due to the small boxsize, it somewhat underpredicts the SFRD for $z < 7$. Our simulation also matches the observed $z=6$ galaxy luminosity function well, which gives us some confidence that our ionizing souce population is representative of the real universe. However a substantial fraction of our ionizing flux comes from sources that are too faint to be observed; we defer a discussion of this topic to Paper III in this series (So et al., {\em in prep.})

Our goal was not to predict the precise redshift of ionization completion, as this would depend on details such as escape fraction of ionizing radiation from galaxies and their stellar populations that we do not model directly. Rather our goal was to examine the mechanics of reionization in its early, intermediate, and late phases within a model which is calibrated to the observed source population. Nonetheless, we present a model in which reionization completes at $z\approx 6$, consistent with observations. 

At early and intermediate times we find that reionization proceeds ``inside-out", confirming the results of many previous investigations \citep{Gnedin2000,RazoumovEtAl2002,SokasianEtAl2003,FurlanettoEtAl2004,IlievEtAl2006,TracCen2007,TracEtAl2008}. However, at late times isolated islands of neutral gas are ionized from the outside-in as they have no internal sources of ionization. Even this characterization is somewhat oversimplified when {\em degree of ionization} is considered, as we discussed in Sec. \ref{IOOI}. It accurately depicts how reionization proceeds for a low degree of ionization (> 0). However for high degrees of ionization, ``inside-out-middle" is more appropriate, as filaments lag behind low and high density regions, as discussed by \cite{FinlatorEtAl2009}. 

Our most interesting findings concerns the widely used analytic model of reionization introduced by \cite{MadauEtAl1999}. Both the instantaneous (Equation \ref{eq:ndot}) and time-dependent (Equation \ref{eq:dQdt}) versions of this model underpredict the time (overpredict the redshift) when reionization completes, when applied to our simulation. There are two reasons for this having to do with the detailed mechanics of reionization at early and late times respectively. At early times, I-fronts are propagating in regions of higher density than the cosmic mean since the first sources are highly biased. Higher densities translate into slower bubble expansion rates, retarding $Q_{\hii}(z)$ relative to a solution which assumes the cosmic mean density (Figure \ref{Qeffv2}). At late times, which we loosely define as $Q_{\hii} > 0.5$, conversion of ionizing photons into new ionized hydrogen atoms becomes inefficient. This can be seen by forming this ratio directly from the simulation data (Figure \ref{Ndot_Ratio}), or by defining a global \hi  ionization parameter (Equation \eqref{eq:IP} and Figure \ref{IP}). The consequence of this dropping ionization efficiency, which is as low as 0.05 at overlap in our simulation, is to further retard $Q_{\hii}(z)$ relative to a solution which assumes an ionization efficiency of unity (Figure \ref{Qeffv3}).

We have introduced a modified version of \cite{MadauEtAl1999}'s time-dependent analytic reionization model in Equation \eqref{eq:dQdtdbg}. Modifications which correct for the above-mentioned effects apply to the source term only, {\em not to the recombination term}. These corrections are therefore totally independent of issues like clumping factors and the temperature of the IGM, which enter into the characteristic recombination time of the IGM. The modifications are introduced as correction factors to the mean density of baryons in the vicinity of ionizing sources at early times ($\delta_b$), and the conversion efficiency of ionizing photons emitted to \hi photoionization rate at late times ($\gamma$). Fits of these two correction factors versus $Q_{\hii}$ are presented in Figures \ref{deltabvsQfit5} and \ref{RatiovsQfit} for consumption by other researchers. At this point we do not know how general these results are. However we have indications based on another simulation we have analyzed with a softer source SED that the functional forms are representative of this class of reionization model. 

The significance of these results to high redshift galaxy observers is the following. Setting $Q_{\hii} = 1$ and $\delta_b = 1$ in Equation \eqref{eq:dQdtdbg}, we derive
\begin{equation}
\dot{n}_{ion} = \frac{1}{\gamma} \frac{\bar{n}_H}{\bar{t}_{rec}}.
\label{eq:ndotgamma}
\end{equation}
This differs from the usual expression used to assess whether a given ionizing photon injection rate can maintain an ionized IGM by the factor $1/\gamma$, which is a factor of $\sim 20$ at overlap in our simulation. If this result is correct, then it means that the required UV luminosity density to maintain an ionized IGM has been underestimated by a factor of approximately 20. However, a more precise statement would be that the UV luminosity density required to maintain the IGM in a {\em highly ionized state; $f_n =10^{-5}$} is 20 times higher than what has been previously estimated. Lower levels of UV luminosity density than that specified in Equation \eqref{eq:ndotgamma} could still maintain the IGM in an ionized state, but one with a higher neutral fraction. 

As we showed in Figure \ref{treceffhubble}, the effective recombination time at and after overlap in our model is comparable to the Hubble time, whether we use the Madau formula to evaluate it for reasonable values for the clumping factor, or we evaluate it directly from our simulation data. This fact casts in doubt the entire instantaneous photon counting argument which is the basis of Equation \ref{eq:ndot}, and the equation becomes less useful for the purposes to which it has been applied (e.g., Robertson et al. 2013). It means that the ionization state of the IGM has a memory on the timescale of $\bar{t}_{rec}$ which is always a significant fraction of $t_{Hubble}$ before overlap, and of order the Hubble time after overlap. We therefore recommend observers use the time-dependent version Equation \eqref{eq:dQdtdbg} in future assessments of high redshift galaxy populations and their role in reionization. 

\subsection{Limitations of the Simulation}

We conclude this section with a brief discussion of the known limitations of our simulation and a comparison of our results with others in the published literature. First the limitations. The principal limitation is the use of a uniform grid, which prevents us from resolving processes occuring inside galaxy halos. The main defect this introduces is an inability to calculate the ionizing escape fraction directly, as is done in some high resolution simulations; e.g., \cite{WiseCen2009,FernandezShull2011}. In our simulation, we calibrate our star formation recipe to match the observed SFRD, and then use that that to calculate UV feedback cell-by-cell via Equation \eqref{eq:emissivity}. We use a value for $\epsilon_{UV}$ taken from \cite{RicottiEtAl2002} for an unattenuated low metallicity stellar population. We underestimate the amount of internal attenuation of ionizing flux due to our limited resolution within halos, and we do not incorporate an explicit escape fraction parameter in Equation \eqref{eq:emissivity}. Effectively, we assume $f_{esc}(ISM)=1$. Using a lower value for $f_{esc}$ would result in a lower overlap redshift \citep{PetkovaSpringel2011a}. Clearly, it would be desirable to vary this parameter in future studies. 

A second limitation of our simulation is that we have presented only one realization in a relatively small box. Previous studies have shown that \hii bubbles reach a characteristic size of $\sim 10$ Mpc comoving in the lates stages of reionization \citep{FurlanettoEtAl2004,ZahnEtAl2007,ShinEtAl2008}. At 20 Mpc on a side, our box is scarcely larger than this. Therefore one can ask how robust our results are to boxsize. We have addressed this by carrying out a simulation of identical physics, spatial, and mass resolution in a volume 64 times as large as the one described in this paper. The simulation is carried out in a box 80 Mpc on a side on a uniform mesh of $3200^3$ cells, and with an equivalent number of dark matter particles. Results of this simulation will be presented in a forthcoming paper (So et al., in preparation). For the present we merely state that the $Q_{\hii}(z)$ curve for the $800^3$ simulation falls within the $\pm 1 \sigma$ band for the larger simulation, where this band is obtained by subdividing the large simulation into 64 cubes of size 20 Mpc on a side, and calculating the mean and standard deviation. While the larger box begins to ionize at a slightly earlier redshift, due to the presence of higher sigma peaks forming galaxies, both simulations complete reionization at the same redshift, $z_{reion} = 5.8$. The $Q_{\hii}(z)$ curve for the $800^3$ simulation is near the lower edge of the band, which means that at intermediate redshifts ($7 \leq z \leq 8$), where the difference is largest, the small box simulation underestimates the fraction of the volume that is ionized by about 20\%, with differences smoothly decreasing to lower and higher redshift. 

A third limitation is that our SFRD systematically deviates from observations below $z \sim 7$, flattening and then decreasing slightly, rather than continuing to rise (Figure \ref{SFR}). The large box simulation does not show this effect, but rather tracks the observed SFRD over the entire range of redshifts. The difference in the mean SFRD between the large and small box simulations increases smoothly from 0.1 dex at $z=9$ to 0.3 dex at $z=6$. The higher levels of star formation in the large box simulation accounts for the higher ionized volume fraction at intermediate redshifts. Nonetheless, the two simulations complete reionization at virtually the same redshift, which is a curious result which we address in a subsequent paper.

Another limitation of our method is the use of flux-limited diffusion (FLD) to transport radiation. It is well known that FLD does not cast shadows behind opaque blobs. This could potentially overestimate how rapidly the IGM ionizes, and hence overestimate $z_{reion}$. In Paper I we showed through a direct comparison between FLD and an adaptive ray tracing method incorporated in the {\em Enzo} code on a standard test problem that the differences in the volume- and mass-weighted ionized volume fraction are small. This was for a rather small volume with a small number of ionizing sources. The differences will likely be even smaller as larger volumes containing larger numbers of sources are considered. At the present time, no fully-coupled radiation hydrodynamic simulations of reionization using ray tracing in large volumes are available to compare our method against, to confirm or deny this conjecture. 

\subsection{Comparison with Other Self-Consistent Simulations}

Finally, we compare our results to the results of several recent fully-coupled simulations of reionization including hydrodynamics, star formation, and radiative transfer. \cite{PetkovaSpringel2011a} simulated a (10 Mpc/h)$^3$ volume with the {\tt Gadget-2} code coupled to a variable tensor Eddington factor moment method for the ionizing radiation field sourced by star forming galaxies. They carried out a suite of simulations with $2 \times 128^3$ gas and dark matter particles, varying the ionizing escape fraction and the mean energy per photon from hot, young stars. The also performed one simulation at $2 \times 256^3$ resolution to check for convergence. Our simulation has 80/10 times superior mass resolution as their $128^3/256^3$ simulations. Because {\tt Gadget} is a Lagrangian code, our Eulerian simulation has 8/16 times lower resolution in the highest density regions, but 4.46/2.23 times higher resolution at mean density, and even higher resolution compared to the {\tt Gadget} simulations in low density voids. Our method also has a more accurate adaptive subcycling timestepping scheme for the coupled radiation-ionization-energy equations, obviating the need to model nonequilibrium effects by means of a gas heating parameter $\epsilon$. 

Morphologically, our results are qualitatively similar, as are the neutral hydrogen fraction versus overdensity phase diagrams. As might be expected from the two methods, the phase diagrams show some differences at the highest and lowest overdensities which is likely a resolution effect. The SFRD in the \cite{PetkovaSpringel2011a} simulation is about an order of magnitude higher than observed, making a direct comparison on $Q_{\hii}(z)$ somewhat problematic. However, since they vary the ionizing escape fraction, we can roughly compare their $f_{esc}=0.1$ case with our results. Their model completes reionization at $z \approx 5$ compared to our own which completes at $z \approx 5.8$. They plot the quantity $log[1-Q_{\hii}(z)]$, which makes the end of reionization look abrupt. We plot $Q_{\hii}(z)$, which makes the end of reionization look slow. When we plot $log[1-Q_{\hii}(z)]$ using our data, it looks very similar to their curves, and shows a rapid plunge in the average neutral fraction at late times.  \cite{PetkovaSpringel2011a} do not compare with the predictions of the \cite{MadauEtAl1999} model, nor do they investigate the evolution of clumping factors, recombination times, or the number of photons per H atom to achieve overlap as we do. We do not investigate the properties of the $z=3$ IGM via Lyman $\alpha$ forest statistics, as they do. Therefore further comparisons are not possible at this time. 

\cite{FinlatorEtAl2012} examined some of the same issues we have, hence a comparison with their results is informative. They carried out a suite of {\tt Gadget-2} simulations in small volumes (3, 6)Mpc/h coupled to a variable tensor Eddington factor moment method. Unlike \cite{PetkovaSpringel2011a}, the radiation transport is solved on a uniform Cartesian grid, rather than evaluated using the SPH formalism. The results presented in \cite{FinlatorEtAl2012} use $2 \times 256^3$ dark matter and gas particles, which given their small volumes, yields a similar mass resolution to our simulation, superior spatial resolution in high density regions, and slightly coarser spatial resolution at mean density and below. However, their radiation transport is done on coarse $16^3$ mesh, which in their fiducial run is $536$ comoving kpc $\approx 20 \times$ as coarse as ours. Their simulation thus coarse-grains the radiation field relative to the density field, which necessitates the introduction of a sub (radiation) grid model for unresolved self-shielded gas (i.e., Lyman limit systems). The effect of their subgrid model is to remove some gas in the overdensity regime $1 \leq \Delta_b \leq 50$ in the calculation of the \hii clumping factor, thereby lowering it. Since our radiation field is evolved on the same grid as the density field, we have not included an explicit subgrid model for unresolved self-shielded gas. Lyman limit systems, with neutral column densities of $\sim 10^{17}$ cm$^{-2}$, have a characteristic size of 10 physical kpc \citep{Schaye2001,McQuinnEtAl2011}. At $z=6$ this is 70 comoving kpc, which is resolved by 3 grid cells in our simulation. While this is lower than one would ideally like (5-10 cells), we believe we can make an apples-to-apples comparison between our resolution-matched simulation results and Finlator et al.'s results. 

Our results are in broad agreement with those of \cite{FinlatorEtAl2012}, with some minor quantitative differences.  We both find that the unthresholded baryon clumping factor $C_b$ significantly overestimates the clumping in ionized gas at redshifts approaching overlap, and therefore that it should not be used to estimate the mean recombination rate in the IGM. We confirm their findings that properly accounting for the ionization state and temperature of gas of moderate overdensities lowers the clumping factor to less than $\approx 6$ (in our case less than 5; see Figure \ref{ClumpingFactors}).  Finlator et al. quote a value for $C_{\hii}$ of 4.9 at $z=6$ taking self-shielding into account, which is in good agreement with our value of $C_{tt\hii} \approx 4.8$. However, they favor a lower value for $C$ of 2.7-3.3 taking temperature corrections into account. This can be compared with our value for $C_{RR} \approx 2.3$, which includes temperature corrections but also excludes gas with $\Delta_b<1$. Including this low density gas, as Finlator et al. do, would raise this value somewhat since a larger range of densities enter into the average. We conclude therefore that clumping factors derived from our simulation are in good agreement with those reported by \cite{FinlatorEtAl2012}. 

We find that approximately 2 photons per hydrogen atom ($\gamma/H\approx 2$) are required to reionize gas satisfying $\Delta_b<100$--our proxy for the fluctuating IGM. \cite{FinlatorEtAl2012} quote a model-dependent value for $\gamma/H$ which depends on the redshift at which the IGM becomes photoheated and thereby Jeans smoothed  (their Fig. 7). For $z=6$,  $\gamma/H \approx 5$, significantly higher than our number evaluated directly from the simulation. However, for $z=8$, when our box is already significantly ionized,  $\gamma/H \approx 3$. Because there are many model-dependent assumptions that go into the Finlator et al. estimate, we consider this reasonably good agreement. However we point out that our estimate is the first to be derived from a self-consistent simulation of reionization with no subgrid models aside from the star formation/radiative feedback recipe. 

Finally, \cite{FinlatorEtAl2012} compare $Q_{\hi}(z)=1-Q_{\hii}(z)$ for their fiducial model with the time-dependent model of \cite{MadauEtAl1999}. They point out the sensitivity of the redshift of overlap on the choice of clumping factor, which enters into the recombination time, and showed that $C_{\hii}$ provides better agreement with theory at early times than $C_b$, consistent with our findings. Since small discrepancies in $Q_{\hii}(z)$ at early times are masked by plotting $Q_{\hi}(z)$, Finlator et al. did not discover the need for our overdensity correction $\delta_b$. Similar to us, they found that even with the best clumping factor estimate the analytic model predicts that reionization completes earlier than the simulation by $\Delta z \approx 1$. They ascribe this delay to finite speed-of-light effects (which can only account for $\Delta z =0.1$), while we ascribe it to nonequilibrium ionization effects. \cite{FinlatorEtAl2012} did not propose modifications to the \cite{MadauEtAl1999} model to improve agreement with simulation, as we do in Equation \eqref{eq:dQdtdbg}.

%\item simulation assumptions
%\item what we have found
%\item comparison with previous work (agreement/disagreement)
%\item implications for connecting high-z galaxies to reionization
%\item areas for improvement

%\end{itemize}