\documentclass[letterpaper,10pt]{article}
\usepackage{geometry}   % See geometry.pdf to learn the layout
                        % options.  There are lots.
\usepackage[latin1]{inputenc}
\usepackage{graphicx}
\usepackage{epstopdf}
\usepackage{amsmath,amsfonts,amssymb}

\DeclareGraphicsRule{.tif}{png}{.png}{`convert #1 `dirname #1`/`basename #1 .tif`.png}

\author{Daniel R. Reynolds}
\title{{\tt DualFLD}: \\
A FLD-based X-ray and UV Radiation Solver for ENZO}

\renewcommand{\(}{\left(}
\renewcommand{\)}{\right)}
\newcommand{\vb}{{\bf v}_b}
\newcommand{\xvec}{{\bf x}}
\newcommand{\Omegabar}{\bar{\Omega}}
\newcommand{\rhob}{\rho_b}
\newcommand{\dt}{\Delta t}
\newcommand{\Eot}{E^{OT}}
\newcommand{\Ef}{E_f}
\newcommand{\sighat}{\hat{\sigma}}
\newcommand{\Fnu}{{\bf F}_{\nu}}
\newcommand{\Pnu}{\overline{\bf P}_{\nu}}
\newcommand{\R}{I\!\!R}
\newcommand{\Rthree}{\R^3}
\newcommand{\eh}{e_h}
\newcommand{\ec}{e_c}
\newcommand{\Edd}{\mathcal F}
\newcommand{\Eddnu}{\Edd_{\nu}}
\newcommand{\mn}{{\tt n}}
\newcommand{\mB}{\mathcal B}
\newcommand{\mC}{{\mathcal C}}
\newcommand{\mL}{{\mathcal L}}
\newcommand{\mD}{{\mathcal D}}
\newcommand{\mDnu}{\mD_{\nu}}
\newcommand{\mCnu}{\mC_{\nu}}
\newcommand{\mLnu}{{\mathcal L}_{\nu}}
\newcommand{\mCe}{\mC_e}
\newcommand{\mLe}{\mL_e}
\newcommand{\mCn}{\mC_{\mn}}
\newcommand{\mLn}{\mL_{\mn}}


\textheight 9truein
\textwidth 6.5truein
\addtolength{\oddsidemargin}{-0.25in}
\addtolength{\evensidemargin}{-0.25in}
\addtolength{\topmargin}{-0.5in}
\setlength{\parindent}{0em}
\setlength{\parskip}{2ex}


\begin{document}
\maketitle

\section{Introduction}
\label{sec:intro}

This document describes a new, highly scalable,
field-based dual X-ray and UV radiation solver for Enzo, 
{\tt DualFLD}.  In this solver, we transport up to two radiation
fields, hereafter named $E_{Xr}$ and $E_{UV}$.  Both of these are
assumed to be scalar-valued fields defined throughout the spatial
domain, and are evolved using a flux-limited diffusion (FLD)
approximation for the propagation, absorption and dilution of
radiation in a cosmologically-expanding spatial domain.  Each
radiation field is assumed to have either a monochromatic SED at a
specified input frequency, or is treated as an integrated radiation
energy density with an assumed radiation spectrum.  Separate SED
choices may be made for both $E_{Xr}$ and $E_{UV}$.  Interaction
between these radiation fields is assumed to occur only through
interaction with the baryonic density and total energy fields defined
in the volume, i.e. there is no direct X-ray to UV radiation coupling.
The interactions with the baryonic densities and total energy fields
occurs through four mechanisms:
\begin{itemize}
\item[(a)] Absorption of radiation due to the monochromatic or 
  frequency-integrated opacities valid over the frequency spectrum
  where each field is valid.
\item[(b)] Photo-ionization of chemistry fields due to the radiation 
  present within each spatial cell.
\item[(c)] Photo-heating of the total energy field due to radiation
  present within each spatial cell.
\item[(d)] Secondary ionizations of chemistry fields due to X-ray
  radiation at frequencies exceeding 100 eV.
\end{itemize}
While the {\tt DualFLD} solver handles the propagation of both
radiation fields, it does not handle steps (b)-(d) directly, and
instead computes photo-ionization and photo-heating rates that are
passed to Enzo's built-in chemical ionization and gas heating
solvers.

The defining characteristics of this solver in comparison
with the {\tt gFLDSplit} solver are three-fold.  First, similarly to
{\tt gFLDSplit}, the {\tt DualFLD} solver propagates radiation in a
separate operator-split step from the chemistry and heating, though in
this solver the radiation solve itself is split into separate UV and
X-ray solves (since they are assumed to be non-interacting), and the
photo-heating and photo-ionization rates are computed afterwards from
the combined radiation contributions.  Second, the {\tt DualFLD}
solver itself {\em cannot} perform ionization or heating directly,
instead relying on Enzo's existing solvers for those physics.  Third,
the {\tt DualFLD} module {\em only} allows a standard
chemistry-dependent model (i.e. it does not allow the {\em local
thermodynamic equilibrium} model that is supported by {\tt
gFLDSplit}).  

This guide is designed to highlight the solvers and equations
available in this module.  Additional details on the model derivation,
numerical methods and verification tests will be provided in
subsequent publications.  In order to best aid in usage of the {\tt
DualFLD} solver, we first provide details on how to use the solver
within an Enzo simulation, and follow the description of those
parameters with the supporting equations and algorithm description.



\section{{\tt DualFLD} usage}
\label{sec:module_usage}

In order to use the {\tt DualFLD} radiation solver module, and to
allow optimal control over the solver methods used, there are a number
of parameters that may be supplied to Enzo.  We group these into two
categories, those associated with the general startup of the module
via the Enzo infrastructure, and those that may be supplied to the
{\tt DualFLD} module itself.  However, prior to embarking on a
description of these parameters, there are a few requirements for any
problem that wishes to use the {\tt DualFLD} module. 

The Enzo source code is located in the subdirectory {\tt enzo/} of the
code repository.  Within this directory, the {\tt Makefile} must be
edited to specify how the code should be run, and where to look for
machine-specific compilation instructions.  The main parameter
names within this {\tt Makefile} are self-explanatory, but the
following four parameters are used to determine the machine-specific
file to use in compilation:
\begin{itemize}
\item {\tt ENZO\_SYSTEM} -- this should hold a description of the
  machine itself (e.g. ``jaguar'')
\item {\tt ENZO\_COMPILER} -- this should hold a description of the
  compilers to use (e.g. ``pgi'')
\item {\tt ENZO\_MODE} -- this should hold either ``mpi'' or
  ``hybrid'', corresponding to the parallelization strategy (pure MPI
  or hybrid MPI/OpenMP, respectively)
\item {\tt ENZO\_SPEC} -- I'm not sure what this should be used for;
  in all cases I've seen it holds the word ``test''
\end{itemize}
With these definitions, the Enzo {\tt Makefile} will look for
machine/compiler/parallelism specific definitions within a file
\begin{verbatim}
   Make.${ENZO_SYSTEM}_${ENZO_COMPILER}.${ENZO_MODE}_${ENZO_SPEC}
\end{verbatim}
located in the {\tt enzo/Macros/} subdirectory of the overall code
repository.  For example, with the definitions {\tt
ENZO\_SYSTEM=reynolds}, {\tt ENZO\_COMPILER=intel}, {\tt
ENZO\_MODE=mpi}, {\tt ENZO\_SPEC=test}, the {\tt Makefile} will read
the relevant compiler/library definitions from the file {\tt
Make.reynolds\_intel.mpi\_test}. 

Within this machine-specific file, definitions must be included for
the parameters:
\begin{itemize}
\item {\tt FC} -- MPI-enabled Fortran compiler
\item {\tt DEFINES} -- Fortran pre-processor definitions
\item {\tt FC\_OPT} -- Fortran compiler optimization flags
\item {\tt CC} -- MPI-enabled C compiler
\item {\tt CC\_DEF} -- C pre-processor definitions
\item {\tt CC\_OPT} -- C compiler optimization flags
\item {\tt CCxx} -- MPI-enabled C++ compiler
\item {\tt CCxx\_DEF} -- C++ pre-processor definitions
\item {\tt CCxx\_OPT} -- C++ compiler optimization flags
\item {\tt LDR} -- Program for linking the final executable
\item {\tt LDR\_OPT} -- Pre-processor definitions for the linker
\item {\tt LIB} -- Library locations and files to link with ({\bf
  must} include HYPRE, HDF5 and SZIP)
\item {\tt INCLUDE} -- Include directories containing relevant header
  files ({\bf must} include HYPRE and HDF5 header file locations)
\item {\tt OPTIONS} -- unknown (currently empty)
\item {\tt OPT} -- combined compilation options for generating {\tt
  .o} files; currently contains {\tt -c \$\{OPTIONS\} \$\{INCLUDE\}}
\item {\tt DEBUG} -- unknown (currently empty)
\end{itemize}

The HYPRE library linked in must be version 2.8.0b or newer.  If a
user must compile HYPRE themselves to obtain this version, they should
make note of the HYPRE configuration option {\tt
--with-no-global-partition}, which must be used for solver scalability
when using over $\sim\!1000$ processors, but which results in slower
executables on smaller-scale problems. 

{\bf Final note:} the SVN repository contains a file {\tt
enzo/mpi\_save.h}.  On most systems, this file is not needed to build
Enzo, and may be ignored.  However, on some systems Enzo may not
compile with the native {\tt mpi.h} file, in which case this file
should be renamed to {\tt enzo/mpi.h}.  It is suggested that you first
attempt to build Enzo with the file retaining the name {\tt
enzo/mpi\_save.h}.  If compilation fails due to MPI-related error
messages, you may then rename the file the {\tt enzo/mpi.h} and try
compilation again.
 


\subsection{Startup parameters}

In a user's main problem parameter file, the following parameters
must be set (their default values are in brackets):
\begin{itemize}
\item {\tt RadiationHydrodynamics} [0] -- this specifies to enable one
  of the FLD solver modules within Enzo's operator-split time-stepping
  approach.  Allowable values include:
  \begin{itemize}
  \item 0 -- disable all FLD solvers
  \item 1 -- enable FLD solvers, use along-side other Enzo physics modules
  \item 2 -- enable FLD solvers and disable all other Enzo physics modules
  \end{itemize}
\item {\tt ImplicitProblem} [0] -- this specifies the type of FLD
  solver to use.  The {\tt DualFLD} solver corresponds to {\tt
  ImplicitProblem = 4}.
\item {\tt ProblemType} [0] -- as usual, this is problem-dependent.
  However, for FLD-based solvers in this version of Enzo, the value of
  ProblemType is typically within the 200's.
\item {\tt RadiationFieldType} [0] -- this can be any value {\em except}
  10 or 11, since those use pre-existing background radiation
  approximations.
\item {\tt RadiativeCooling} [0] -- a nonzero value enables Enzo's
  built-in chemistry/heating solvers.
\item {\tt Multispecies} [0] -- when {\tt RadiativeCooling} is set to
  1, a nonzero value for {\tt Multispecies} will turn on Enzo's
  chemical ionization solvers.  The particular value is used to
  enable various chemistry models, as described in the main Enzo
  documentation, 
\item {\tt RadHydroParamfile} [NULL] -- this should contain the filename
  (with path relative to this parameter file) that contains all
  module-specific solver parameters (discussed below).  While the 
  {\tt DualFLD} module parameters may be supplied in the main
  parameter file, that filename must still be specified here (though
  it is not recommended, since the {\tt ReadParameterFile.C} routine
  will complain about all of the `unknown' parameters that are read
  elsewhere).
\item {\tt CoolDataParameterFile} [NULL] -- this should contain the
  filename (with path relative to this parameter file) that contains
  all CoolData-specific solver parameters.  Typical non-default values
  include 
  \begin{itemize}
  \item {\tt HydrogenFractionByMass} [0.76] -- this gives the fraction of
    total mass comprised of Hydrogen; it is useful for performing Hydrogen-only
    simulations.
  \item {\tt RateDataCaseBRecombination} [0] -- a nonzero value
    denotes that Enzo should use case-B recombination rates instead of
    the default case-A rates.
  \end{itemize}
\end{itemize}

In addition, if a user wishes to set up a new {\tt ProblemType} that
uses the {\tt DualFLD} module, they must allocate standard Enzo
baryon fields having the {\tt FieldType} set to {\tt RadiationFreq0}
and {\tt RadiationFreq1}.  It is these baryon fields that will be
evolved by the {\tt DualFLD} module, with {\tt RadiationFreq0}
corresponding to the X-ray radiation field and {\tt RadiationFreq1}
corresponding to the UV radiation field. If radiative cooling and
ionization are to be performed, additional baryon fields must be allocated
with {\tt FieldType} set to {\tt kphHI} and {\tt PhotoGamma}, as well
as {\tt kphHeI} and {\tt kphHeII} if Helium is included, and 
{\tt kdissH2I} if {\tt Multispecies}>1.  All of these fields may be
initialized to zero.

Furthermore, the {\tt DualFLD} module currently computes its own
emissivity fields $\eta_{Xray}(\xvec)$ and  $\eta_{UV}(\xvec)$, in the
routine {\tt DualFLD\_RadiationSource.src90}.  A user may edit this file
to add in an emissivity field corresponding to their own {\tt
ProblemType}.  Alternatively, a user may separately fill in the  
{\tt BaryonField}s {\tt Emissivity0} and {\tt Emissivity1} in other
Enzo routine(s).  Then, when Enzo is compiled using the pre-processor
directive {\tt EMISSIVITY} and run with the global parameter {\tt
StarMakerEmissivity $\ne$ 0}, the {\tt DualFLD} module will look to
these {\tt BaryonField}s for emissivity values instead of computing
its own. 



\subsection{Module parameters}

Once a user has enabled the {\tt DualFLD} module, they have
complete control over a variety of internal module parameters.  The
parameters are given here, with their default values specified in
brackets, and references to the appropriate equations elsewhere in
this document.
\begin{itemize}
\item {\tt DualFLDXrayOnly} [0] -- this parameter disables propagation
  of the UV radiation field in the solver, and eliminates all
  $E_{UV}$-related contributions to photo-heating and photo-ionization.
\item {\tt DualFLDXrayStatic} [0] -- this parameter disables
  propagation of the X-ray radiation field in the solver, effectively
  freezing the $E_{Xr}$ field to its initial configuration.  This does
  not currently deplete the radiation density due to cosmological
  expansion, although such an enhancement would not be difficult to
  add to {\tt DualFLD} in the future.
\item {\tt DualFLDUVStatic} [0] -- this parameter disables
  propagation of the UV radiation field in the solver, effectively
  freezing the $E_{UV}$ field to its initial configuration.  This does
  not currently deplete the radiation density due to cosmological
  expansion, although such an enhancement would not be difficult to
  add to {\tt DualFLD} in the future.
\item {\tt DualFLDXraySpectrum} [-1] -- this parameter chooses the type of
  assumed X-ray radiation energy spectrum from equation \eqref{eq:spectrumXr}.
  Allowed values include
  \begin{itemize}
  \item[-1.] monochromatic spectrum.
  \item[0.] power law spectrum,
    \[
      \chi(\nu) = \left(\frac{\nu}{\nu_{HI}}\right)^{-1.5}
    \]
  \item[1.] $T=10^5$ K blackbody spectrum, 
    \[
       \chi(\nu) = \frac{8 \pi h
         \left(\frac{\nu}{c}\right)^3}{\exp\left(\frac{h\nu}{k_b 10^5}\right)-1}.
    \]
  \end{itemize}
\item {\tt DualFLDUVSpectrum} [-1] -- this has the same options as
  above, but pertain to the UV radiation field \eqref{eq:spectrumUV}.
\item {\tt DualFLDXrayFrequency} [500] -- if the X-ray spectrum is
  monochromatic, this input specifies the frequency of the radiation
  field (in eV).
\item {\tt DualFLDUVFrequency} [13.6] -- if the X-ray spectrum is
  monochromatic, this input specifies the frequency of the radiation
  field (in eV).
\item {\tt DualFLDChemistry} [3] -- this parameter controls how many
  chemical species the solver should interact with (i.e.~for opacity
  calculations and photo-ionization and photo-heating rates).
  Allowable values are 
  \begin{itemize}
  \item[0.] no chemistry
  \item[1.] Hydrogen chemistry
  \item[3.] Hydrogen+Helium chemistry
  \end{itemize}
\item {\tt DualFLDHFraction} [0.76] -- this parameter controls the
  fraction of baryonic matter comprised of Hydrogen that the solver
  module should assume. Allowable values are $0 \le {\tt
    RadHydroHFraction} \le 1$.
\item {\tt DualFLDMaxDt} [$10^{20}$] -- this parameter sets the value of
  $\dt_{\text{max}}$ from section \ref{sec:dt_selection}; it must be
  greater than 0.  This value is provided in {\em scaled} time units,   
  i.e.~$\dt_{\text{physical}} \le \dt_{\text{max}}*\text{TimeUnits}$, 
  where TimeUnits is Enzo's internal time scaling factor for the
  simulation.
\item {\tt DualFLDMinDt} [0] -- this parameter sets the value of
  $\dt_{\text{min}}$ from section \ref{sec:dt_selection}; it must be
  non-negative.  This value must also be given in scaled time
  units.
\item {\tt DualFLDInitDt} [$10^{20}$] -- this parameter sets the
  initial time step size for the {\tt DualFLD} module.  We note that
  since the module will take the smaller of $\dt_{\text{FLD}}$ and
  $\dt_{\text{Enzo}}$, the default value is never actually used.  This
  value must also be given in scaled time units.
\item {\tt DualFLDDtNorm} [2] -- this parameter sets the value
  of $p$ from equation \eqref{eq:time_error}.  
  \begin{itemize}
  \item A value of $0$ implies to use the $\max$ norm, 
  \item A value $>0$ implies to use the corresponding $p$-norm,
  \item Values $<0$ are not allowed (reset to the default).
  \end{itemize}
\item {\tt DualFLDDtGrowth} [1.1] -- this gives the maximum time step size
  growth factor per solver step (i.e.~allows an increase of 10\% per step).
\item {\tt DualFLDDtXrayFac} [$10^{20}$] -- this gives the value
  of $\tau_{\text{tol}}$ for the variables $E_{Xr}$ from equation
  \eqref{eq:time_estimate}.  They must be positive; the default
  essentially specifies no restrictions on $\dt_{\text{FLD}}$. 
\item {\tt DualFLDDtUVFac} [$10^{20}$] -- this gives the value
  of $\tau_{\text{tol}}$ for the variables $E_{UV}$ from equation
  \eqref{eq:time_estimate}.
\item {\tt DualFLDXrayScaling} [1] -- this gives the scaling 
  factor $s_{Xr}$ from \eqref{eq:variable_rescaling}; supplied values
  must be positive. 
\item {\tt DualFLDUVScaling} [1] -- this gives the scaling 
  factor $s_{UV}$ from \eqref{eq:variable_rescaling}; supplied values
  must be positive. 
\item {\tt DualFLDTheta} [1] -- this parameter specifies the
  value of $\theta$ in equation \eqref{eq:radiation_PDE_theta};
  requires $0\le\theta\le 1$.
\item {\tt DualFLDXrBoundaryX0Faces}, {\tt DualFLDXrBoundaryX1Faces},
  {\tt DualFLDXrBoundaryX2Faces} [0 0] --  these specify the
  boundary-condition types from section \ref{sec:boundary_conditions}
  to use on the lower and upper boundaries in each direction for the
  X-ray radiation field.  Allowable values are
  \begin{itemize}
  \item[0.] periodic (must match on both faces in a given direction)
  \item[1.] Dirichlet
  \item[2.] Neumann
  \end{itemize}
\item {\tt DualFLDUVBoundaryX0Faces}, {\tt DualFLDUVBoundaryX1Faces},
  {\tt DualFLDUVBoundaryX2Faces} [0 0] --  these specify the
  boundary-condition types from section \ref{sec:boundary_conditions}
  to use on the lower and upper boundaries in each direction for the
  UV radiation field.  Allowable values are
  \begin{itemize}
  \item[0.] periodic (must match on both faces in a given direction)
  \item[1.] Dirichlet
  \item[2.] Neumann
  \end{itemize}
\item {\tt DualFLDSolToleranceXray} [$10^{-8}$] -- this parameter
  specifies the linear tolerance $\delta$ from section
  \ref{sec:rad_solve} for the X-ray solve.  Allowable values must be
  between $10^{-15}$ and 1. 
\item {\tt DualFLDSolToleranceUV} [$10^{-8}$] -- this parameter
  specifies the linear tolerance $\delta$ from section
  \ref{sec:rad_solve} for the UV solve.  Allowable values must be
  between $10^{-15}$ and 1. 
\item {\tt DualFLDMaxMGItersXray} [50] -- this positive parameter
  specifies the maximum number of multigrid iterations to perform for
  each radiation field in the MG-CG solver for the X-ray field from section
  \ref{sec:rad_solve}.
\item {\tt DualFLDMaxMGItersUV} [50] -- this positive parameter
  specifies the maximum number of multigrid iterations to perform for
  each radiation field in the MG-CG solver for the UV field from section
  \ref{sec:rad_solve}.
\item {\tt DualFLDMaxPCGItersXray} [50] -- this positive parameter
  specifies the maximum number of preconditioned conjugate gradient
  iterations to perform for each radiation field in the MG-CG solver
  for the X-ray field from section \ref{sec:rad_solve}.
\item {\tt DualFLDMaxPCGItersUV} [50] -- this positive parameter
  specifies the maximum number of preconditioned conjugate gradient
  iterations to perform for each radiation field in the MG-CG solver
  for the UV field from section \ref{sec:rad_solve}.
\item {\tt DualFLDMGRelaxTypeXray} [1] -- this parameter specifies the
  relaxation method used by the multigrid solver for the X-ray field:
  \begin{itemize}
  \item[0.] Jacobi
  \item[1.] Weighted Jacobi
  \item[2.] Red/Black Gauss-Seidel (symmetric)
  \item[3.] Red/Black Gauss-Seidel (nonsymmetric)
  \end{itemize}
  For more information, see the HYPRE user manual.
\item {\tt DualFLDMGRelaxTypeUV} [1] -- this parameter specifies the
  relaxation method used by the multigrid solver for the UV field.
  The values match those above.
\item {\tt DualFLDMGPreRelaxXray} [1] -- this positive parameter
  specifies the number of pre-relaxation sweeps the multigrid solver
  should use in the MG-CG solver for the X-ray field from section
  \ref{sec:rad_solve}. 
\item {\tt DualFLDMGPreRelaxUV} [1] -- this positive parameter
  specifies the number of pre-relaxation sweeps the multigrid solver
  should use in the MG-CG solver for the UV field from section
  \ref{sec:rad_solve}. 
\item {\tt DualFLDMGPostRelaxXray} [1] -- this positive parameter
  specifies the number of post-relaxation sweeps the multigrid solver
  should use in the MG-CG solver for the X-ray field from section
  \ref{sec:rad_solve}. 
\item {\tt DualFLDMGPostRelaxUV} [1] -- this positive parameter
  specifies the number of post-relaxation sweeps the multigrid solver
  should use in the MG-CG solver for the UV field from section
  \ref{sec:rad_solve}. 
\end{itemize}




\section{Flux-limited diffusion radiation model}
\label{sec:rad_model}

We begin with the equation for flux-limited diffusion radiative
transfer in a cosmological medium \cite{ReynoldsHayesPaschosNorman2009},
\begin{equation}
\label{eq:radiation_PDE}
  \partial_{t} E_i + \frac1a \nabla\cdot\(E_i\vb\) =
    \nabla\cdot\(D_i\,\nabla E_i\) - \frac{\dot{a}}{a} E_i - c\kappa_i E_i + \eta_i,
\end{equation}
where here the comoving radiation energy density field $E_i$, emissivity
$\eta_i$ and opacity $\kappa_i$ are functions of space and time, and
where $i\in\{\text{Xr},\text{UV}\}$.  In this equation, the
frequency-dependence of the respective radiation energy density field
has been integrated away, under the premise of an assumed radiation
energy spectrum, 
\begin{align}
  \notag
  & E_{\nu}(\nu,\xvec,t) = \tilde{E}_{Xr}(\xvec,t) \chi_{Xr}(\nu) +
     \tilde{E}_{UV}(\xvec,t) \chi_{UV}(\nu), \\ 
  \notag
  \Rightarrow & \\
  \label{eq:spectrumXr}
  & E_{Xr}(\xvec,t) = \int_{0}^{\infty} E_{\nu}(\nu,\xvec,t)\,\mathrm{d}\nu 
    = \tilde{E}_{Xr}(\xvec,t) \int_{0}^{\infty}
    \chi_{Xr}(\nu)\,\mathrm{d}\nu, \\
  \label{eq:spectrumUV}
  & E_{UV}(\xvec,t) = \int_{0}^{\infty} E_{\nu}(\nu,\xvec,t)\,\mathrm{d}\nu 
    = \tilde{E}(\xvec,t) \int_{0}^{\infty} \chi_{UV}(\nu)\,\mathrm{d}\nu,
\end{align}
where $\tilde{E}_{Xr}$ and $\tilde{E}_{UV}$ are intermediate
quantities (for analysis) that are never computed, and where we have
assumed that the two spectra $\chi_{Xr}(\nu)$ and $\chi_{UV}(\nu)$ do
not overlap (i.e.~$\chi_{Xr}(\nu)$ disappears in the interval
$[0,\nu_{1})$ and $\chi_{UV}(\nu)$ disappears in the
interval $[\nu_{1},\infty)$).  We note that if either assumed spectrum
is the Dirac delta function, $\chi_i(\nu) = \delta_{\nu_i}(\nu)$, then
$E_i$ is a monochromatic radiation energy density at the ionization
threshold $h\nu_i$, and the $-\frac{\dot{a}}{a}E$ term in equation
\eqref{eq:radiation_PDE}, obtained through integration by parts of the
redshift term 
$\frac{\dot{a}}{a}\partial_{\nu}E_{\nu}$, is omitted from
\eqref{eq:radiation_PDE}. Similarly, the emissivity functions
$\eta_i(\xvec,t)$ relate to the true emissivity 
$\eta_{\nu}(\nu,\xvec,t)$ by the formulas
\begin{equation}
\label{eq:emissivity}
  \eta_{Xr}(\xvec,t) =
  \int_{0}^{\infty}\eta_{\nu}(\nu,\xvec,t)\,\mathrm{d}\nu \\
\notag
  \eta_{UV}(\xvec,t) =
  \int_{0}^{\infty}\eta_{\nu}(\nu,\xvec,t)\,\mathrm{d}\nu.
\end{equation}

Within equation \eqref{eq:radiation_PDE}, the function
$D_i$ is the {\em flux limiter} that depends on face-centered values of
$E_i$, $\nabla E_i$ and the opacity $\kappa_i$ \cite{Morel2000},
\begin{align}
  D_i &= \min\left\{c \(9\kappa_{i,f}^2 + R^2\)^{-1/2}, D_{max}\right\},\quad\mbox{and}\quad
  R = \max\left\{\frac{|\partial_{x} E_i|}{E_{i,f}},R_{min}\right\}.
\end{align}
Here the spatial derivative within $R$ is computed using
non-dimensional units at the computational face adjoining two
neighboring finite-volume cells, $D_{max}=0.006\,c\,L_{unit}$ and
$R_{min}=10^{-20}/L_{unit}$ with $L_{unit}$ the length
non-dimensionalization factor for the simulation, and the 
face-centered radiation energy density and opacity are computed using
the arithmetic and harmonic means, respectively,
\[
   E_{i,f} = \frac{E_{i,1} + E_{i,2}}{2}, \qquad
   \kappa_{i,f} = \frac{2\kappa_{i,1} \kappa_{i,2}}{\kappa_{i,1} + \kappa_{i,2}},
\]
where here $E_{i,1}$ and $E_{i,2}$ are the two values of $E_i$ in the
cells adjacent to the face.  Among the many available limiter
formulations we have tested
\cite{HayesNorman2003,Morel2000,ReynoldsHayesPaschosNorman2009}, this  
version performs best at producing causal radiation propgation
speeds in the low-opacity limit typical of reionization simulations.  




\section{Model couplings}
\label{sec:chem_model}

In general, radiation calculations in Enzo are used in simulations
where chemical ionization states are important.  For these situations, 
we couple the radiation equation \eqref{eq:radiation_PDE} with
equations for both the conservation of gas energy and primordial
chemistry ionization/recombination, 
\begin{align}
  \label{eq:cons_energy}
  \partial_t e + \frac1a\vb\cdot\nabla e &=
    - \frac{2\dot{a}}{a}e
    - \frac{1}{a\rhob}\nabla\cdot\left(p\vb\right) 
    - \frac1a\vb\cdot\nabla\phi + G - \Lambda  + \dot{e}_{SF}, \\
  \label{eq:chemical_ionization}
  \partial_t \mn_j + \frac{1}{a}\nabla\cdot\(\mn_j\vb\) &=
    \alpha_{j,k} \mn_e \mn_k - \mn_j \Gamma_{j}^{ph}, \qquad
    j\in\{\text{HI, HII, HeI, HeII, HeIII}\}. 
\end{align}
Here, $\mn_{j}$ is the comoving number density for each chemical
species, $\mn_k$ corresponds to chemical species that interact with
species $\mn_j$, and $\mn_e$ is the electron number density.  In these
equations, all terms are evolved by Enzo's built-in chemistry and gas
energy solvers, though some of the relevant rates result from
radiation-dependent couplings. Specifically, the gas can be
photo-heated by the radiation through the term 
\begin{align}
  \label{eq:G}
  G &= \frac{c\,E_{UV}\,\sum_{j} \mn_j
    \int_{\nu_{j}}^{\infty} \sigma_{j}\, \chi_{UV}
    \left(1-\frac{\nu_{j}}{\nu}\right)\,
    d\nu}{\rhob\,\int_{0}^{\infty} \chi_{UV} d\nu}
  + \frac{Y_{\Gamma}\,c\,E_{Xr}\,\sum_{j} \mn_j
    \int_{\nu_{j}}^{\infty} \sigma_{j}\, \chi_{Xr}
    \left(1-\frac{\nu_{j}}{\nu}\right)\,
    d\nu}{\rhob\,\int_{0}^{\infty} \chi_{Xr} d\nu},
\end{align}
for $j\in\{\text{HI,HeI,HeII}\}$, where the X-ray secondary
photo-heating coefficient $Y_{\Gamma}$ depends on the electron
fraction $\xi$ in a cell via the formula 
\begin{equation}
\label{eq:Ygamma}
  Y_{\Gamma} = 0.9971 \left[1 - \left(1-\xi^{0.2663}\right)^{1.3163}\right].
\end{equation}
Within the Enzo code base, $G$ is stored in the baryon field
{\tt PhotoGamma}, for communication between {\tt DualFLD} and Enzo's
heating/cooling solvers.

Additionally, the photo-ionization rates $\Gamma_{j}^{ph}$ within
equation \eqref{eq:chemical_ionization} depend on the X-ray and UV
radiation fields via the formulas
\begin{align}
\label{eq:Gamma}
  \Gamma_j^{ph} &= \frac{Y_{j}\, c\, E_{Xr}\, \int_{\nu_j}^{\infty}
    \frac{\sigma_j(\nu) \chi_{Xr}(\nu)}{\nu}\,\mathrm d\nu}{h\,
    \int_{0}^{\infty} \chi_{Xr}(\nu)\,\mathrm d\nu} 
  + \frac{c\, E_{UV}\, \int_{\nu_j}^{\infty}
    \frac{\sigma_j(\nu) \chi_{UV}(\nu)}{\nu}\,\mathrm d\nu}{h\,
    \int_{0}^{\infty} \chi_{UV}(\nu)\,\mathrm d\nu}.
\end{align}
In this formula, we employ the X-ray photo-ionization coefficients
\begin{align}
\label{eq:YH}
  Y_{HI} &= 0.3908 \left(1 - \xi^{0.4092}\right)^{1.7592}, \\
  Y_{HeI} &= 0.0554 \left(1 - \xi^{0.4614}\right)^{1.666}, \\
  Y_{HeII} &= 0.
\end{align}
Within the Enzo code base, the rates $\Gamma_{HI}^{ph}$,
$\Gamma_{HeI}^{ph}$ and $\Gamma_{HeII}^{ph}$ are held in the baryon fields
{\tt kphHI}, {\tt kphHeI} and {\tt kphHeII} for communication between
{\tt DualFLD} and Enzo's chemistry solvers.

Lastly, the frequency-integrated opacities depent on the chemical
state at each spatial location,
\begin{align}
\label{eq:opacityXr}
  \kappa_{Xr} \ = \ \frac{\sum_{j} 
  \mn_{j} \int_{\nu_{j}}^{\infty} \chi_{Xr}\,\sigma_{j}\,d\nu}{
  \int_{0}^{\infty} \chi_{Xr}\,d\nu}, \quad j\in\{\text{HI,HeI,HeII}\}
  \\
\label{eq:opacityUV}
  \kappa_{UV} \ = \ \frac{\sum_{j} 
  \mn_{j} \int_{\nu_{j}}^{\infty} \chi_{UV}\,\sigma_{j}\,d\nu}
  {\int_{0}^{\infty} \chi_{UV}\,d\nu}, \quad j\in\{\text{HI,HeI,HeII}\},
\end{align}
where these integrals with the assumed radiation spectra $\chi_{Xr}(\nu)$
and  $\chi_{UV}(\nu)$ handle the change from the original
frequency-dependent radiation equation to the integrated grey
radiation equations.


Within the {\tt DualFLD} module, the baryon field {\tt kdissH2I}
is always set to 0. 





\section{Numerical solution approach}
\label{sec:solution_approach}

We evolve these models in an operator-split fashion, wherein we solve
the radiation equations \eqref{eq:radiation_PDE} separately from the 
gas energy correction and chemistry equations
\eqref{eq:cons_energy} and \eqref{eq:chemical_ionization}, 
which are evolved together.  These solves are coupled to Enzo's
existing operator-split solver framework in the following manner:
\begin{itemize}
\item[(i)] Project the dark matter particles onto the finite-volume
  mesh to generate a dark-matter density field;
\item[(ii)] Solve for the gravitational potential and compute the
  gravitational acceleration field;
\item[(iii)] Evolve the hydrodynamics equations using an
  up-to-second-order accurate explicit method, and have the velocity
  $\vb$ advect both the X-ray and UV radiation fields, $E_{Xr}$ and
  $E_{UV}$; 
\item[(iv)] Evolve the coupled gas energy correction and chemistry
  evolution equations;  
\item[(v)] Advect the dark matter particles with the Particle-Mesh
  method;
\item[(vi)] Evolve the radiation fields implicitly in time using an
  up-to-second-order accurate method.
\end{itemize}

The implicit solution approach for step (vi) is similar to the one
from \cite{NormanReynoldsSoHarkness2013}; here we describe only enough
to point out the available user parameters, and more fully describe
some additional options available in the solver.  

In solving the steps (vi) we use a method of lines approach for the
space-time discretization of \eqref{eq:radiation_PDE}, in that we
first discretize the equations in space using a second-order-accurate,
uniform-grid, finite volume discretization, and then evolve the
resulting system of ODEs in time.


\subsection{Radiation subsystem}
\label{sec:rad_solve}

Assuming that all spatial derivatives are treated using standard
second-order centered difference approximations on our finite-volume
grid, we need only discuss the time-discretization of our radiation
equation \eqref{eq:radiation_PDE}.  Our approach follows a standard
two-level $\theta$-method, 
\begin{align}
  \label{eq:radiation_PDE_theta}
  E_i^n - E_i^{n-1} &- \theta\dt\left(\nabla\cdot\(D_i^{n-1}\,\nabla E_i^n\)
    - \frac{\dot{a}}{a} E_i^n - c\kappa_i^n E_i^n + \eta_i^n\right) \\ 
  \notag
  & - (1-\theta)\dt\left(\nabla\cdot\(D_i^{n-1}\,\nabla E_i^{n-1}\) -
    \frac{\dot{a}}{a} E_i^{n-1} - c\kappa_i^{n-1} E_i^{n-1} +
    \eta_i^{n-1}\right) = 0, 
\end{align}
where the parameter $0\le\theta\le 1$ defines the time-discretization,
and where we have assumed that the advective portions of \eqref{eq:radiation_PDE}
have already been taken care of through Enzo's hydrodynamics solvers.
Recommended values of $\theta$ are 1 (backwards Euler) and $\frac12$
(trapezoidal, a.k.a.~Crank-Nicolson).  

Whichever $\theta$ value we use (as long as it is nonzero), the
equation \eqref{eq:radiation_PDE_theta} is linearly-implicit in the
time-evolved radiation energy density $E_i^n$.  We write this in
predictor-corrector form (for ease of boundary condition
implementation), which we will write as
\begin{align}
\label{eq:linear_system}
  J s = b, \qquad E_i^n = E_i^{n-1} + s.
\end{align}
We approximately solve this linear equation for the update $s$,
to a tolerance $\delta$,
\begin{align}
\label{eq:linear_system_approx}
  \| J s - b \|_2 \le \delta,
\end{align}
using using a multigrid-preconditioned conjugate gradient iteration.  

Both the X-ray and UV fields are solved using the same
time-discretization parameter $\theta$, and the same multigrid solver
parameters {\tt DualFLDSolTolerance}, {\tt DualFLDMaxMGIters}, 
{\tt DualFLDMGRelaxType}, {\tt DualFLDMGPreRelax} and 
{\tt DualFLDMGPostRelax}; however, it would be trivial to extend the
current {\tt DualFLD} solver to allow specification of different
values for each radiation field.





\subsection{Time-step selection}
\label{sec:dt_selection}

Time steps are chosen adaptively in an attempt to control error in the
calculated solution.  To this end, we first define an heuristic
measure of the time accuracy error in a radiation field $E_i$ as
\begin{align}
\label{eq:time_error}
  err = \left(\frac1N \sum_{j=1}^N
    \left(\frac{E_{i,j}^{n}-E_{i,j}^{n-1}}{\omega_j}\right)^p\right)^{1/p}, 
\end{align}
where the weighting vector $\omega$ is given by
\begin{align}
\label{eq:time_weighting}
  \omega_j &= \sqrt{E_{i,j}^n E_{i,j}^{n-1}} + 10^{-3}, \quad j=1,\ldots,N.
\end{align}
i.e.~we scale the radiation change by the geometric mean
of the old and new states, adding on a floor value of $10^{-3}$ in case any
of the states are too close to zero.  This approach works well when
the internal solution variables are unit-normalized, or at least close
to unit-normalized, since the difference between the old and new
solutions, divided by this weighting factor $\omega$, should give a
reasonable estimate of the number of significant digits that are
correct in the solution. 

With these error estimates \eqref{eq:time_error} for both $E_{Xr}$ and
$E_{UV}$, we set the new time step size for each subsystem based on
the previous time step size and a user-input tolerance $\tau_{\text{tol}}$ as
\begin{align}
\label{eq:time_estimate}
  \dt^{n} = \frac{\tau_{\text{tol}} \dt^{n-1}}{err}.
\end{align}
Since $E_{Xr}$ and $E_{UV}$ are evolved separately, we allow either
solver to subcycle at a faster rate if necessary to allow convergence
of the underlying linear solver.  However, in general we enforce that
both fields utilize the same step size,
\begin{align}
\label{eq:FLD_time_estimate}
  \dt^{n} &= \min\{\dt_{Xr}^{n},\dt_{UV}^{n},\dt_{Enzo}^{n}\},
\end{align}
where $\dt_{\text{Enzo}}$ is the time step size that Enzo's other
routines (e.g.~hydrodynamics) would normally take.  We further note
that the {\tt DualFLD} solver module will force Enzo to similarly take
this more conservative time step size, due to the tight physical
coupling between radiation transport and chemical ionization.  

Additionally, a user may override these adaptive time step controls
with the input parameters $\dt_{\text{max}}$ and $\dt_{\text{min}}$.
However, even with such controls in place the overall time step will
still be selected to adhere to the bound required by Enzo's other
physical modules, i.e.
\begin{align}
  \dt^{n} &= \min\{\dt_{\text{min}}^{n},\dt_{Enzo}^{n}\}.
\end{align}




\subsection{Variable rescaling}
\label{sec:variable_rescaling}

In case Enzo's standard unit non-dimensionalization using 
{\tt DensityUnits}, {\tt LengthUnits} and {\tt TimeUnits} is
insufficient to render the resulting solver values $E_{Xr}$ and
$E_{UV}$ to have nearly unit magnitude, the user may input additional
variable scaling factors to be used inside the {\tt DualFLD} module.
The basic variable non-dimensionalization of these fields is to create
a non-dimensional radiation field value by dividing the physical value
(in ergs/cm$^3$) by the factor {\tt DensityUnits * LengthUnits$^2$ *
TimeUnits$^{-2}$}.  As would be expected, the values of $E_{Xr}$ and
$E_{UV}$ may differ by orders of magnitude, so it is natural that they
should be non-dimensionalized differently.

To this end, if we denote these user-input values as $s_{Xr}$, and
$s_{UV}$, then the {\tt DualFLD} module defines the rescaled variables 
\begin{align}
\label{eq:variable_rescaling}
  \tilde{E}_{Xr} = E_{Xr} / s_{Xr}, \qquad \tilde{E}_{UV} = E_{UV} / s_{UV},
\end{align}
and then uses the rescaled variables $\tilde{E}_{Xr}$ and
$\tilde{E}_{UV}$ in its internal routines instead of Enzo's
``non-dimensionalized'' internal
variables $E_{Xr}$ and $E_{UV}$.  If the user does not know
appropriate values for these scaling factors {\em a-priori}, a
generally-applicable rule of thumb is to first run their simulation
for a small number of time steps and investigate Enzo's HDF5 output
files to see the magnitude of the values stored internally by Enzo; if
these are far from unit-magnitude, appropriate scaling factors
$s_{Xr}$ and $s_{UV}$ should be supplied in the {\tt DualFLD}
parameter input file.




\subsection{Boundary conditions}
\label{sec:boundary_conditions}

As the radiation equation \eqref{eq:radiation_PDE} is parabolic,
boundary conditions must be supplied on the radiation field $E_i$.  The
{\tt DualFLD} module allows three types of boundary conditions to
be placed on the radiation field:
\begin{itemize}
\item[0.] Periodic,
\item[1.] Dirichlet, i.e.~$E_i(x,t) = g(x), \; x\in\partial\Omega$, and
\item[2.] Neumann, i.e.~$\nabla E_i(x,t)\cdot n = g(x), \; x\in\partial\Omega$.
\end{itemize}
In most cases, the boundary condition types (and values of $g$) are
problem-dependent.  When adding new problem types, these conditions
should be set near the bottom of the file {\tt DualFLD\_Initialize.C}, 
otherwise these will default to either (a) periodic, or (b) will use
$g=0$, depending on the user input boundary condition type.



\section{Concluding remarks}
\label{sec:conclusions}

We wish to remark that the module is not large (one header
file, 14 C++ files, 2 F90 files), and all files begin with the 
{\tt DualFLD} prefix.  While we have strived to ensure that the
module is bug-free, there is still work to be done in enabling
additional physics, including more advanced time-stepping interactions
with the rest of Enzo (especially when ionization sources ``turn on''
abruptly), and adaptive or static mesh refinement.  

Feedback/suggestions to are welcome.


\bibliography{sources}
\bibliographystyle{siam}
\end{document}
